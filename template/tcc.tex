%
% exemplo genérico de uso da classe iiufrgs.cls
% $Id: iiufrgs.tex,v 1.1.1.1 2005/01/18 23:54:42 avila Exp $
%
% This is an example file and is hereby explicitly put in the
% public domain.
%
\documentclass[tc,oneside]{tcc}

\usepackage[T1]{fontenc}        % pacote para conj. de caracteres correto
\usepackage{float}
\usepackage[chapter]{minted}
\usepackage[utf8]{inputenc}     % pacote para acentuação
\usepackage{graphicx}           % pacote para importar figuras
\usepackage{times}              % pacote para usar fonte Adobe Times
\usepackage{mathptmx}           % p/ usar fonte Adobe Times nas fórmulas
\usepackage{setspace}
\usepackage{listings}
\usepackage{hyperref}
\usepackage{subfigure}
\usepackage{array}
\usepackage{tabu}
\usepackage{placeins}
\lstset{numbers=left,
stepnumber=1,
firstnumber=1,
numberstyle=\tiny,
extendedchars=true,
lineskip=3pt,
frame=tb,
basicstyle=\footnotesize,
stringstyle=\ttfamily,
showstringspaces=true
}

\newmintedfile[jsoncode]{json}{
frame=single,
breaklines,
breakanywhere,
tabsize=2
}

\newmintedfile[jscode]{javascript}{
frame=single,
linenos,
breaklines,
breakanywhere,
tabsize=2
}

\newmintedfile[htmlcode]{html}{
frame=single,
linenos,
breaklines,
breakanywhere,
tabsize=2
}

\usepackage{appendix}			% pacote para usar o environment appendices (precisa usar por causa do pdfpages)
\usepackage{pdfpages}			% pacote para incluir PDFs inteiros (soh funciona com pdflatex)

\renewcommand\listingscaption{Código-fonte}
\renewcommand\listoflistingscaption{Lista de Códigos-fonte}

\newcommand{\apud}[4]{(\mkbibparens{\citeyear{#1},\space{#2}, apud \citeauthor{#3}, \citeyear{#3}, \space{#4}})}

%
% Informações gerais do TRABALHO DE CONCLUSÃO
%
\title{Título Do Anteprojeto}

\author{Ciclano}{Fulano da Silva}

\advisor[Prof.~Dr.]{Silva}{João José}
\coadvisor[Prof.~Dr.]{Silva}{José João}

%defina aqui os nomes dos convidados das bancas, para compor a folha de avaliacao
\def\primeiroconvidadobanca{ Prof. Dr. Homer Simpson }%
\def\segundoconvidadobanca{Prof. Dr. Jack Sparrow}


% a data deve ser a da defesa; se nao especificada, são gerados
% mes e ano correntes
%\date{maio}{2001}


% palavras-chave
% iniciar todas com letras minúsculas, exceto no caso de abreviaturas
\keyword{formatação eletrônica de documentos}
\keyword{\LaTeX}
\keyword{ABNT}


% inicio do documento
\begin{document}

\maketitle

\renewcommand{\titlepagespecificinfo}{Monografia apresentada como requisito parcial para obtenção do t\'itulo de Tecn\'ologo em Sistemas para Internet.}

%\makefacepage

\clearpage

% Inclui a folha de avaliacao. Essa folha NAO deve estar presente na primeira versao entregue para a banca,
% apenas para a versao final do texto, apos aprovacao. Mantenha o comando a seguir comentado ate a geracao
% da versao final.
%\makegradepage

%epigrafe
%\input{pretextuais/epigrafe.tex}

% agradecimentos
%\chapter*{Agradecimentos}

Agradeço ao \LaTeX\ por facilitar muito a vida de quem precisa escrever um texto academico\ldots

%resumo
% resumo no idioma do documento

\begin{abstract}
Este documento é um exemplo de como formatar o trabalho de conclusão para o TSI do IFSul Câmpus Charqueadas. \emph{O texto do resumo não deve conter mais do que 500 palavras.}
\end{abstract}

%abstract
% resumo no segundo idioma
% como parametros devem ser passados o titulo e as palavras-chave
% no outro idioma, separados por vírgulas

\begin{englishabstract}{BiteUp: an application for macronutrient control}{macronutrients, nutrition, machine learning, food tracking, well-being}

With the growing interest in nutrition and well-being, there is an increasing demand for digital tools to assist in food tracking. Despite the availability of various technological solutions, many people still face difficulties in adequately monitoring their daily macronutrient intake, compromising the achievement of their nutritional goals. In this context, this project proposes the development of a mobile application that allows the user to log consumed foods and track, in real time, both the nutritional impact and the subjective effects of these food choices on their well-being.

The application performs automatic calculation of the amount of macronutrients consumed and remaining, facilitating daily meal planning and promoting adherence to healthier habits. Its main differentiator lies in the ability to analyze the user's individual response after each meal, transcending simple caloric and nutritional logging. Through a recurring feedback system, users can report aspects related to their post-prandial state, such as energy level, mood, disposition, and possible physical discomforts.

The collected data are processed to identify behavioral and nutritional patterns, enabling inferences about which foods contribute to positive or negative sensations. With the accumulation of information, the system becomes progressively more accurate in its recommendations, dynamically adapting to each user's individual needs and preferences. In addition to nutritional monitoring functionalities and food inventory, the solution incorporates machine learning techniques to offer an adaptive and personalized experience.

The project also considers essential challenges related to the accuracy of nutritional information, interface usability, and personal data security. Thus, it seeks to develop an efficient, reliable tool focused on promoting a diet that effectively contributes to the physical and emotional balance of individuals.

\end{englishabstract}

% lista de figuras
\listoffigures

% lista de tabelas
\listoftables

% lista de codigos-fonte.
\listoflistings

% lista de abreviaturas e siglas
% o parametro deve ser a abreviatura mais longa
\begin{listofabbrv}{ANVISA}
        \item[ALCOA] Acrônimo relacionado à integridade de dados.
        \item[ANVISA] Agência Nacional de Vigilância Sanitária
        \item[BPx] Boas Práticas (onde ``x'' pode representar diferentes áreas, como Fabricação BPF, Laboratório BPL, etc.).
        \item[EMA] \textit{European Medicines Agency} --- Agência Europeia de Medicamentos
        \item[FDA] \textit{Food and Drug Administration} --- Administração de Alimentos e Medicamentos
        \item[IFSul] Instituto Federal Sul-rio-grandense
        \item[PIC/S] \textit{Pharmaceutical Inspection Co-operation Scheme} --- Esquema de Cooperação em Inspeção Farmacêutica
        \item[PWA] \textit{Progressive Web App}
        \item[RDC] Resolução de Diretoria Colegiada
        \item[TSI] Tecnólogo em Sistemas para Internet
        \item[UML] \textit{Unified Modeling Language} --- Linguagem de Modelagem Unificada
\end{listofabbrv}


% sumario
\tableofcontents

% aqui comecam as secoes e o texto propriamente dito


\chapter{Introdução}
\label{ch:introducao}

A busca por uma alimentação equilibrada e personalizada tem se tornado progressivamente mais relevante, impulsionada pelo crescente interesse em nutrição e bem-estar. Apesar dessa consciência ampliada, muitas pessoas ainda encontram dificuldades significativas para monitorar o consumo de macronutrientes ao longo do dia, comprometendo assim o alcance de seus objetivos nutricionais. Entre os principais obstáculos identificados destacam-se a falta de tempo, limitações financeiras e a dificuldade em modificar hábitos alimentares profundamente enraizados \cite{almeida2020, fernandes2019}.

Embora exista amplo conhecimento científico sobre os impactos da alimentação na saúde, e apesar da disponibilidade de aplicativos destinados ao acompanhamento nutricional, observa-se que muitos usuários relatam dificuldades em manter o controle consistente de suas refeições. A ciência já esclareceu por meio de diversas pesquisas quais alimentos trazem benefícios e quais podem ser prejudiciais quando consumidos em excesso; entretanto, transformar esse conhecimento em hábito diário permanece como um desafio significativo. As aplicações de registro alimentar podem auxiliar nesse processo, porém muitas apresentam limitações que dificultam sua adoção contínua, tais como interfaces pouco intuitivas, excesso de funcionalidades pouco práticas, ausência de personalização efetiva e escassa atenção ao bem-estar subjetivo do usuário após o consumo das refeições.

Consequentemente, identifica-se um problema residual relevante: mesmo com acesso à informação e a ferramentas tecnológicas, persiste a dificuldade de muitos indivíduos em monitorar o consumo de macronutrientes de forma simples e ajustar suas escolhas alimentares com base no modo como cada refeição afeta seu organismo. Adicionalmente, as aplicações mais populares focam quase exclusivamente em calorias e macronutrientes, negligenciando a percepção individual de cada usuário sobre energia, disposição, humor ou desconfortos, fatores que possuem forte impacto na adesão a hábitos mais saudáveis.

Nesse contexto, o desenvolvimento de uma aplicação voltada ao acompanhamento diário da ingestão de macronutrientes, aliada a um sistema de \textit{feedback} sobre o bem-estar do usuário, configura-se como uma solução promissora. A proposta visa desenvolver uma aplicação que permita ao usuário registrar os alimentos consumidos ao longo do dia e acompanhar, em tempo real, o impacto desses alimentos em sua meta nutricional. A aplicação realizará automaticamente o cálculo da quantidade de macronutrientes disponíveis e consumidos, facilitando o planejamento alimentar e incentivando a adesão a uma dieta equilibrada.

Para tornar o processo mais dinâmico e intuitivo, a aplicação contará com uma API responsável por acessar um banco de dados de alimentos pré-cadastrados, fornecendo informações detalhadas sobre os macronutrientes de cada item alimentício. Adicionalmente, o usuário terá a possibilidade de cadastrar novos alimentos no inventário, garantindo maior flexibilidade e personalização no acompanhamento de sua dieta.

O principal diferencial da aplicação reside na capacidade de analisar a resposta individual do usuário após o consumo de cada refeição. Essa funcionalidade transcende o simples registro calórico e nutricional, oferecendo uma camada adicional de inteligência fundamentada no bem-estar subjetivo do usuário. Por meio de um sistema de \textit{feedback} recorrente, os usuários poderão informar como se sentiram após as refeições, avaliando aspectos como nível de energia, humor, disposição e eventuais desconfortos físicos.

Com essa abordagem, espera-se que a aplicação contribua significativamente para a adesão a hábitos alimentares mais saudáveis, proporcionando aos usuários uma ferramenta eficiente e intuitiva para o monitoramento de sua alimentação diária.

\section{Objetivo}
\label{sec:objetivo}

O objetivo deste trabalho relaciona-se diretamente à busca por uma solução eficaz para o problema identificado: a dificuldade enfrentada por muitos usuários ao monitorar de forma prática e personalizada sua ingestão diária de macronutrientes, especialmente quando possuem metas nutricionais específicas, como o ganho de massa muscular. Com base nisso, a aplicação visa atuar como facilitador no alcance dos objetivos nutricionais do usuário, proporcionando autonomia, organização e suporte contínuo ao longo de sua rotina alimentar de forma prática e rápida, sem comprometer a funcionalidade.

\subsection{Objetivos Específicos}
\label{subsec:objespecificos}

\begin{itemize}
    \item Propor uma solução que promova o acompanhamento nutricional individualizado, considerando critérios de escalabilidade, eficiência e acessibilidade por meio das ferramentas vistas ao longo do curso.

    \item Estudar cálculos e diretrizes nutricionais reconhecidos pela literatura e por profissionais da área, com o objetivo de embasar a solução para o problema identificado com coerência.

    \item Garantir que usuários tenham acesso a um banco de dados de alimentos confiável e flexível, permitindo tanto a consulta a alimentos preexistentes quanto o cadastro de novos, promovendo maior personalização no registro alimentar.

     \item Promover o engajamento do usuário no registro alimentar diário por meio de uma experiência de uso intuitiva e centrada em suas rotinas.

    \item Auxiliar o alcance de metas nutricionais pessoais por meio do acompanhamento automático e contínuo da ingestão de macronutrientes.
    
    \item Permitir que os usuários registrem suas sensações e percepções após as refeições ou ao fim do dia, fornecendo dados subjetivos que possam contribuir para uma futura recomendação personalizada.
  
    \item Avaliar a funcionalidade, usabilidade e adequação da aplicação a partir de testes com usuários, identificando pontos de melhoria que garantam sua efetividade como ferramenta de apoio nutricional.

\end{itemize}

\section{Organização do Trabalho}
\label{sec:organizacao}

Este trabalho está estruturado em capítulos que abordam de forma progressiva os aspectos teóricos, técnicos e práticos da aplicação desenvolvida. A seguir, apresenta-se um resumo do conteúdo de cada capítulo.

O Capítulo~\ref{ch:introducao} apresenta a introdução do trabalho, contextualizando o problema enfrentado por pessoas que buscam monitorar a alimentação com foco em objetivos específicos, como ganho de massa muscular. São apresentados o problema, os objetivos da aplicação e a justificativa para sua construção.

O Capítulo~\ref{ch:referencial_teorico} apresenta a fundamentação teórica que sustenta os principais cálculos adotados no desenvolvimento deste trabalho, com base na literatura científica pertinente à área.

O Capítulo~\ref{ch:trabalhos_relacionados} descreve os principais trabalhos relacionados, incluindo aplicativos já existentes e estudos acadêmicos que abordam o uso de tecnologias na área da nutrição. Com base nessa análise, são destacados os pontos fortes e limitações de cada solução, buscando identificar oportunidades de inovação.

O Capítulo~\ref{ch:solucao_proposta} detalha a proposta do sistema desenvolvido, abordando sua arquitetura, as funcionalidades principais e os diferenciais incorporados, como o sistema de feedback e a recomendação baseada em inteligência artificial. São também justificadas as escolhas tecnológicas utilizadas no projeto.

O Capítulo~\ref{ch:desenvolvimento} tem por objetivo apresentar as etapas práticas da construção do código, trazendo as principais funções desenvolvidas e a lógica por trás das implementações até o momento em que tudo se conecta.

O Capítulo~\ref{ch:conclusoes_trabalhos_futuso} entrega uma visão final dos objetivos previstos e alcançados, além de uma proposta de continuação que foi projetada ao longo do desenvolvimento, com possíveis versões melhoradas das funcionalidades desenvolvidas e também ideias inovadoras que só foram identificadas durante a efetiva realização da proposta.

Além disso, ao longo do texto, são utilizadas imagens, tabelas e trechos de código-fonte para ilustrar e detalhar a construção da solução, contribuindo para a compreensão completa do funcionamento da aplicação.

\chapter{Referencial Teórico}
\label{ch:referencial_teorico}

O referencial teórico apresenta os principais conceitos e diretrizes adotadas para o desenvolvimento da aplicação, fornecendo o embasamento conceitual e técnico necessário à condução deste trabalho. A compreensão dos aspectos relacionados à Integridade de Dados e à Validação de Sistemas Computadorizados é fundamental para garantir que a solução esteja alinhada aos objetivos propostos, bem como às exigências regulatórias estabelecidas para o setor farmacêutico nos quesitos anteriormente citados.

\section{Integridade de Dados}

A integridade de dados no âmbito farmacêutico é uma característica essencial para garantir a qualidade dos produtos e assegurar a conformidade dos processos. O cumprimento desse aspecto é crucial para assegurar a confiabilidade e a consistência das informações ao longo do seu ciclo de vida --- da geração à utilização, retenção e destruição --- conforme apresentado por \citet{vieira2023contribuiccao}.

Inicialmente abordada por meio de Boas Práticas de Fabricação e Boas Práticas de Distribuição, a integridade de dados fez parte da evolução regulatória nas últimas décadas, passando a ser tratada de forma mais ampla, considerando todo o ciclo de vida do produto. Essa mudança trouxe requisitos mais rigorosos para o controle e a segurança das informações, reforçando sua relevância e consolidando-a como um requisito regulatório \cite{rattan2018data}.

Marcos importantes incluem a publicação do documento \textit{Electronic Records; Electronic Signatures – Scope and Application, 21 CFR Part 11}, em 2003, pela agência reguladora FDA dos Estados Unidos. Esse documento teve como objetivo esclarecer o escopo e as implicações deste regulamento, publicado anteriormente em 1997. Ambos abordaram os primeiros conceitos e requisitos relacionados à integridade dos dados. Posteriormente, esses conceitos foram consolidados no acrônimo ALCOA\footnote{Atribuível, Legível, Contemporâneo, Original e Acurado} \cite{vieira2023contribuiccao, FDA2003part11}.

Como parte de um processo de melhoria contínua, esse modelo foi aprimorado em 2010, quando a agência reguladora EMA da União Europeia publicou o documento \textit{Reflection Paper on Expectations for Electronic Source Data and Data Transcribed to Electronic Data Collection Tools in Clinical Trials}. Essa publicação expandiu os princípios do modelo ALCOA, dando origem ao ALCOA+, ao adicionar os atributos Completo, Consistente, Duradouro e Disponível \cite{vieira2023contribuiccao, EMA_INS_GCP_454280_2010}.

Posteriormente, em 2023, a EMA reforçou esses requisitos ao publicar o \textit{Guideline on Computerised Systems and Electronic Data in Clinical Trials}, que introduziu o modelo ALCOA++, conforme apresentado na Tabela~\ref{tab:table_2_1}. Com essa atualização, foi incorporado o atributo Rastreável, ampliando os requisitos necessários para garantir a integridade dos dados \cite{european2023guideline}.

Essa trajetória regulatória consolidou a integridade de dados como um pilar essencial na indústria farmacêutica, sustentando práticas para assegurar a qualidade dos medicamentos e a conformidade dos processos. Os princípios associados, atualmente formalizados no modelo ALCOA++, estabelecem exigências normativas relacionadas à integridade de dados, que as empresas devem atender para garantir a segurança, a eficácia dos medicamentos e o alinhamento com as regulamentações vigentes.

\begin{table}[h]
\centering
\caption{Princípios do ALCOA++.}
\label{tab:table_2_1}
\begin{tabular}{|l|p{11cm}|}
\hline
\textbf{Atributo}   & \textbf{Requerimento} \\ \hline
Atribuível (A)      & Deve ser possível identificar o indivíduo ou sistema informatizado que executou uma atividade registrada e quando isso ocorreu. Isso também se aplica a quaisquer alterações nos registros. \\ \hline

Legível (L)         & Todos os registros devem ser claros e compreensíveis para garantir seu entendimento e utilização. \\ \hline

Contemporâneo (C)   & Evidências de ações, eventos ou decisões devem ser registradas no momento em que ocorrem. \\ \hline

Original (O)        & O registro original pode ser descrito como a primeira captura de informação, seja em papel ou eletronicamente.\\ \hline

Acurado (A)         & Os registros devem refletir fielmente os fatos para garantir sua exatidão. \\ \hline

Completo (+)        & Todas as informações essenciais para reconstituir um evento são indispensáveis para sua compreensão. Um registro eletrônico completo inclui metadados e trilhas de auditoria. \\ \hline

Consistente (+)     & A criação, processamento e armazenamento de informações devem seguir um padrão lógico e coerente. \\ \hline

Duradouro (+)       & Os registros devem ser mantidos de forma segura e acessível durante todo o período exigido. Isso significa que precisam permanecer intactos e legíveis como registros permanentes e duráveis. \\ \hline

Disponível (+)      & Os registros devem estar acessíveis para consulta durante todo o período de retenção exigido. Eles precisam ser legíveis e disponíveis para consulta por auditorias, inspeções e órgãos reguladores. \\ \hline

Rastreável (++)     & Os dados devem permanecer rastreáveis durante todo o seu ciclo de vida, com qualquer alteração devidamente documentada e sem comprometer a informação original. Caso necessário, modificações devem ser explicadas e registradas nos metadados\footnotemark, incluindo a trilha de auditoria. \\ \hline
\end{tabular}
\footnotesize{Fonte: \cite{european2023guideline}.}
\end{table}

\footnotetext{Dados que descrevem os atributos de outros dados, fornecendo contexto e significado.
 }

\section{Validação de Sistemas Computadorizadores}

A validação de sistemas computadorizados é um processo essencial para garantir a confiabilidade de sistemas informatizados em ambientes regulados. Segundo a Resolução RDC Nº 658 \cite{ANVISARDC658}, o termo validação é definido como a ``ação de provar, de acordo com os princípios das Boas Práticas de Fabricação, que qualquer procedimento, processo, equipamento, material, atividade ou sistema realmente leva aos resultados esperados''.

O processo de validação é realizado por meio de atividades de verificação, incluindo testes relacionados aos requisitos de instalação e à avaliação funcional, além da comprovação documentada de todas as atividades executadas. A partir disso, é possível evidenciar que o sistema atende aos requisitos dos usuários e às especificações técnicas do fornecedor, demonstrando que opera conforme suas funcionalidades previstas. Dessa forma, assegura-se o funcionamento correto e seguro, em conformidade com as normas regulatórias aplicáveis e alinhado com as necessidades do usuário \cite{qualidade360}.

De acordo com o Guia nº 33 \cite{guia33ANVISA2025}, a validação de um sistema computadorizado é caracterizada pela obtenção e manutenção da conformidade aos requisitos regulatórios e ao uso pretendido. Isso é alcançado por meio da adoção das atividades pertinentes ao ciclo de vida do sistema, conforme os planos estabelecidos e os resultados obtidos na validação, além da aplicação de controles operacionais. Esse controle é fundamental para assegurar que o sistema mantenha seu estado validado ao longo do tempo.

O ciclo de vida de um sistema computadorizado, em ambiente regulado, deve seguir um fluxo em cascata composto por quatro fases, conforme ilustrado na Figura \ref{fig:2.1}. Cada uma dessas etapas contempla atividades específicas, com o objetivo de garantir a conformidade do sistema desde sua aquisição até sua aposentadoria.

\begin{figure}[!h]
        \centerline{\includegraphics[width=35em]{capitulo2/img/cicloVida.png}}
        \caption{As fases do Ciclo de Vida}
        \label{fig:2.1}
        \centerline{{Fonte: Guia n° 33 \cite{guia33ANVISA2025}}}
\end{figure}

As quatro fases do ciclo de vida de um sistema computadorizado são descritas a seguir:

\begin{itemize}
    \item \textbf{Conceito}: Definição dos requisitos do sistema, avaliação de soluções potenciais e viabilidade do projeto, de modo a garantir que atendam às necessidades do negócio e às exigências regulatórias.
    \item \textbf{Projeto}: Avaliação do impacto quanto às Boas Práticas (BPx), seleção do fornecedor, aquisição e implementação do sistema, validação e liberação para uso. A abordagem adotada para cada sistema é definida com base na avaliação de impacto.
    \item \textbf{Operação}: Utilização do sistema e manutenção do estado validado, com possibilidade de alterações controladas, de modo a assegurar que ele continue operando em conformidade com os requisitos necessários, requisitos iniciais e novas necessidades de negócio.
    \item \textbf{Aposentadoria}: Descontinuação planejada do sistema, incluindo tratativas para preservação, migração ou descarte dos dados do sistema, garantindo a conformidade com requisitos regulatórios.
\end{itemize}

Dentre elas, a fase de Conceito pode ser considerada uma das mais importantes, pois é nela que são definidos os requisitos do sistema, considerando as necessidades do processo, os requerimentos normativos e os aspectos relacionados à gestão de riscos. Essas definições impactam diretamente as demais fases do ciclo de vida do sistema computadorizado. Sob a perspectiva da gestão de riscos e da definição de requisitos, a Instrução Normativa Nº 134 \cite{IN134_2022} estabelece:

\begin{quote}
Art. 8°. A gestão de riscos deve ser aplicada durante todo o ciclo de vida do sistema computadorizado, levando em consideração a segurança do paciente, a integridade dos dados e a qualidade do produto.

Art. 29°. O acesso aos dados armazenados deve ser garantido durante todo o período de armazenamento.

Art. 30°. Devem ser feitos backups de todos os dados relevantes.

Art. 33°. Baseada em análise de risco, deve ser considerada a construção de um sistema de trilha de auditoria de todas as deleções ou alterações relevantes às Boas Práticas de Fabricação.

Art. 36°. Devem existir controles físicos ou lógicos que assegurem que o acesso ao sistema computadorizado é permitido apenas às pessoas autorizadas.
\end{quote}

Dessa forma, os requisitos definidos na fase de Conceito devem ser suficientemente detalhados, contemplando os requisitos regulatórios e as necessidades do negócio, considerando os riscos envolvidos, a fim de garantir a conformidade do projeto e oferecer suporte adequado às atividades subsequentes do ciclo de vida. Entre os requisitos indispensáveis para que um sistema seja considerado passível de validação, destacam-se: o controle de acesso, assegurando que o acesso ao sistema computadorizado seja permitido apenas a pessoas autorizadas e de forma controlada; a trilha de auditoria, garantindo a rastreabilidade das ações e alterações relevantes na aplicação; e a implementação de mecanismos de disponibilidade e integridade, que preservem as informações e assegurem a continuidade operacional mesmo diante de falhas. Tais requisitos estão devidamente fundamentados nos artigos normativos previamente mencionados, garantindo a conformidade regulatória e assegurando a robustez do sistema.
\chapter{Trabalhos Relacionados}
\label{ch:trabalhos_relacionados}

Para a elaboração deste capítulo, foi realizada uma análise de soluções tecnológicas e trabalhos acadêmicos relacionados ao monitoramento nutricional e ao uso de tecnologias digitais para o apoio em práticas alimentares. As fontes consultadas incluem artigos indexados no \textit{Google Scholar}, além da análise de aplicativos amplamente utilizados no mercado.

A busca priorizou publicações dos últimos dez anos e soluções tecnológicas de destaque, com foco em aplicações que ofereçam funcionalidades voltadas ao registro alimentar, recomendação nutricional, integração com dispositivos de rastreamento de atividade física e uso de técnicas de inteligência artificial para personalização das recomendações. Os principais critérios para inclusão foram a relevância da solução no contexto de monitoramento alimentar e o alinhamento com os objetivos do presente trabalho, especialmente no que se refere ao apoio ao usuário na adoção de hábitos alimentares mais saudáveis e ao acompanhamento de metas nutricionais.

Este capítulo apresenta um panorama crítico das soluções analisadas, destacando seus pontos fortes e limitações, de modo a subsidiar o desenvolvimento da aplicação proposta, considerando os aprendizados obtidos com as abordagens existentes.


\section{MyFitnessPal}

MyFitnessPal apresenta-se como uma das principais aplicações nutricionais disponíveis no mercado, com um vasto número de usuários em todo o mundo \cite{myfitnesspal}. O aplicativo oferece uma série de funcionalidades voltadas a indivíduos que buscam alcançar objetivos específicos relacionados à saúde e ao condicionamento físico, como perda de peso, manutenção ou ganho de massa muscular.
Entre os principais recursos da plataforma, destaca-se o contador de calorias, que é configurado com base nas informações fornecidas pelo usuário no momento do cadastro, como peso, altura, idade, sexo e objetivo (perda, manutenção ou ganho de peso). A partir desses dados, o sistema estima automaticamente a ingestão calórica diária recomendada.
Além disso, o app possui uma funcionalidade para acompanhar a ingestão de macros (proteínas, carboidratos e gorduras) ao longo do dia. Contudo, vale ressaltar que o monitoramento mais detalhado dos macronutrientes só está disponível na versão paga, o que pode limitar a experiência de usuários da versão gratuita.

Para registrar um alimento consumido, o usuário deve selecionar uma das opções disponíveis no banco de dados da aplicação, escolher o tamanho da porção e alocar esse alimento em uma das categorias de refeição do dia, como café da manhã, almoço, jantar ou lanches. No entanto, o sistema impede a edição dos valores nutricionais de alimentos previamente cadastrados e obriga o usuário a selecionar uma das categorias preexistentes, não permitindo uma customização mais livre do diário alimentar.

\textbf{Pontos fortes do MyFitnessPal:}
\begin{itemize}
    \item Ampla base de dados de alimentos, com suporte a alimentos industrializados.
    \item Integração com dispositivos de rastreamento de atividade física, como Apple Watch, entre outros.
    \item Registro por escaneamento de código de barras.
        
\end{itemize}


\textbf{Pontos fracos:}
\begin{itemize}
    \item Funcionalidade principal restrita à versão paga, como acompanhamento detalhado de macronutrientes.
    \item Impossibilidade de editar os valores nutricionais de alimentos cadastrados no banco de dados, limitando a precisão do controle.
    \item Obrigatoriedade de escolher categorias predefinidas para refeições, reduzindo a flexibilidade de registro e a particularidade do usuário.
    \item Falta de análise qualitativa e contextual dos alimentos consumidos. O app não oferece, por exemplo, alertas sobre combinações alimentares inadequadas ou efeitos subjetivos pós-refeição.
    \item Interface não amigável, com muita informação na tela e pouco foco no que é relevante. Também não possibilita uma edição de interface para algo mais simplificado.
    
\end{itemize}

\section{Sistema de Raciocínio Baseado em Casos para
Recomendação de Programa Alimentar}

O artigo analisado apresenta o desenvolvimento de um sistema inteligente baseado em técnicas de aprendizado de máquina, utilizando o Raciocínio Baseado em Casos (RBC), com o intuito de auxiliar nutricionistas na tomada de decisões mais assertivas em relação à prescrição de programas alimentares personalizados. A proposta visa oferecer recomendações nutricionais padronizadas, considerando as características individuais de cada paciente, com base em históricos de atendimentos e consultas anteriores.

O funcionamento do sistema baseia-se na comparação de novos casos com registros previamente armazenados, identificando padrões que possam ser reutilizados de forma adaptada. Primeiramente, o sistema realiza a recuperação de casos anteriores que apresentem semelhança com o novo caso em análise. Em seguida, reutiliza as soluções aplicadas anteriormente, realizando as devidas adaptações para atender às especificidades do paciente atual. Após isso, a solução gerada passa por uma etapa de revisão, onde é avaliada para garantir coerência e adequação. Por fim, o novo caso e sua respectiva solução são armazenados no banco de dados, enriquecendo o sistema com mais conhecimento e contribuindo para recomendações futuras.

Essa abordagem permite que o sistema aprenda continuamente com os dados inseridos, tornando-se progressivamente mais eficiente e preciso na geração de recomendações nutricionais personalizadas, ao mesmo tempo em que apoia o profissional nutricionista com base em evidências concretas. \cite{telles_rbc}

Ao adotar uma abordagem semelhante à descrita no artigo, especialmente no uso de raciocínio baseado em casos, a plataforma proposta neste trabalho pode se beneficiar de uma lógica inteligente e adaptativa para recomendação alimentar. Assim como o sistema estudado oferece suporte aos nutricionistas ao aprender com situações passadas, a aplicação em desenvolvimento pode utilizar os registros de consumo alimentar e o \textit{feedback} subjetivo dos usuários (por exemplo, como se sentiram após uma refeição) para construir um histórico capaz de gerar sugestões mais assertivas no futuro. Esse tipo de modelagem contribui para um aprendizado contínuo e personalizado, em que a tecnologia se adapta de maneira dinâmica ao usuário, potencializando os resultados como o ganho de massa muscular por meio de ajustes alimentares baseados em padrões previamente identificados.

\section{Alimente-se}
O alimente-se é um aplicativo voltado para o acompanhamento alimentar diário com foco em facilitar o controle alimentar e criar melhores hábitos, o diferencial do uso é ser um app simples de usar, a interface é amigável e tem técnicas de UI/UX bem definidas já no menu principal.

Outro diferencial é ter a possibilidade de salvar alimentos frequentes com vínculo ao perfil de usuário, com essa função é possível registrar alimentos frequentes com poucos toques, facilitanto a utilização frequente do App com um uso mais fluído para quem tem uma alimentação estruturada e bem definida.

A inteligência artificial também é um dos pontos diferenciais do app, ela promove uma avaliação individual para cada refeição do usuário, destacando observações simples baseadas no objetivo nutricional personalizado de cada usuário, como alto percentual de sódio, baixa ingestão de fibras em um café da manhã por exemplo. \cite{alimentese}

\textbf{Pontos fortes:}
\begin{itemize}
    \item Interface intuitiva que facilita o uso diário do app para usuários com uma rotina bem definida.
    \item Agilidade ao salvar refeições, possibilitando o usuário salvar itens favoritos no perfil para que sejam acessados de forma facilitada em um uso rotineiro.
\end{itemize}

\textbf{Pontos fracos:}
\begin{itemize}
    \item Base de dados mais fraca em relação a outros aplicativos
    \item A avaliação da IA pode ser genérica as vezes, principalmente para usuários com alguma restrição alimentar.
    \item Há recursos funcionais e importantes que não são de possível acesso para usuários da versão gratuita, limitando o usuário e obrigando-o a adquirir a versão paga do aplicativo.
\end{itemize}


\section{Growth - Dieta e Treino}

A aplicação da Growth é voltada para praticantes de musculação e treino de força, oferecendo uma gama de recursos para o acompanhamento nutricional e também recursos voltados ao treino. Um dos principais diferenciais do aplicativo é a possibilidade de acessar a ficha de treino de profissionais renomados patrocinados pela marca, permitindo assim que o usuário se inspire e utilize a ficha de treino dos atletas, fato que apresenta-se como um estímulo à prática das atividades físicas e também uma orientação, já que os usuários têm a opção de acompanhar as técnicas e estratégias que seus ídolos do esporte estão aplicando e utilizar como inspiração no seu próprio treino.

No que diz respeito ao registro de nutrientes, o aplicativo apresenta diversas limitações. A interface é confusa e pouco intuitiva, o que dificulta o uso diário e desmotiva até mesmo os usuários mais pacientes, levando-os a buscar alternativas. Ademais, o banco de dados não é bem estruturado, obrigando o usuário a cadastrar os alimentos a cada utilização, mesmo em caso de itens simples e recorrentes, prática semelhante ao uso de planilhas. Também não há possibilidade de salvar alimentos para uso posterior, o que torna necessário repetir o processo de registro a cada acesso, aumentando o tempo e o esforço despendidos. \cite{growth}

\textbf{Pontos fortes:}
\begin{itemize}
    \item Fichas de treino oficiais dos atletas da Growth
    \item Boa integração com a marca, permitindo visualização de suplementos e demais conteúdos
\end{itemize}

\textbf{Pontos fracos:}
\begin{itemize}
    \item Registro alimentar pouco/nada prático.
    \item Não possui um banco de dados com alimentos básicos pré-cadastrados.
    \item Interface confusa para usuários iniciantes.
    \item Pouca personalização na parte da dieta, comprometendo a utilização da função para objetivos mais específicos.
\end{itemize}

\section{Comparativo}

\begin{table}[H]
    \centering
    \begin{tabular}{|p{3cm}|p{3cm}|p{3cm}|p{3cm}|p{3cm}|}
    
        \hline
             & \textbf{Registro alimentar} & \textbf{Banco de dados} & \textbf{Análise de dados} & \textbf{Menu principal} \\
        \hline

        \hline
            \textbf{Alimente-se} & Muito bom, mas obriga a escolher tipo de refeição & Limitado, mas funciona &     Analisa refeições trazendo insigts com IA & informações importantes na versão paga \\
        \hline

        \hline
            \textbf{Growth} & Confuso e manual & Quase inexistente & Não existe & Muitas informações, cada tipo de refeição ocupa quase meia tela \\
        \hline

        \hline 
            \textbf{MyFitnessPal} & Bom, mas obriga a escolher tipo de refeição, não permite que seja livre & Ótima base de dados & Análise básica na versão paga & Informações de macronutrientes na versão paga \\
        \hline
    
        \hline
            \textbf{Solução Proposta (BiteUp)} & Registro alimentar simplificado com a opção de tipo, não obrigação & Baseado na ideia do MyFitnessPal, com padrão e opção de incremento para os usuários & Análise de dados baseado no \textit{feedback} do usuário & Simplificado a informações mais importantes no menu principal\\
        \hline
    \end{tabular}
    \caption{Comparativo entre aplicações semelhantes}
    \label{tab:comparativo}
\end{table}


A partir da análise de trabalhos relacionados e de aplicativos similares, foi possível identificar diversos aspectos relevantes que contribuem para a compreensão dos diferenciais, limitações e potencialidades das soluções já disponíveis no mercado. Os critérios de avaliação adotados foram relacionados à usabilidade, com ênfase na comparação da facilidade e praticidade em diferentes aspectos, tais como:

\begin{itemize}
    \item Registro de alimentos em cada aplicação.
    \item Existência, funcionalidade, e abrangência dos bancos de dados das aplicações.
    \item Personalização da aplicação para o usuário, com base na análise de dados coletados.
    \item Facilidade de acesso aos itens comumente mais utilizados.
\end{itemize}

Esses critérios serviram de base para a construção de um quadro comparativo entre as soluções analisadas, apresentado na Tabela~\ref{tab:comparativo}, o qual sintetiza os principais pontos observados e permite uma visualização clara das vantagens e limitações de cada alternativa.


No que se refere à avaliação do aplicativo Alimente-se, observa-se que se trata de uma ferramenta bastante promissora. A interface é bem desenvolvida e claramente pensada para dispositivos móveis, com boa usabilidade e organização dos menus. A base de dados de alimentos, embora não seja muito extensa, é funcional para o uso cotidiano. Um ponto positivo é a possibilidade de salvar alimentos para registros futuros, o que otimiza o tempo do usuário. Por outro lado, um ponto de atenção é que as informações detalhadas sobre os macronutrientes só estão disponíveis na versão paga, o que limita a experiência de quem utiliza o app gratuitamente.

Já a aplicação da Growth Supplements apresenta dificuldades significativas em termos de usabilidade. O processo de registro alimentar é bastante complexo e exige que o usuário insira manualmente todos os dados dos alimentos, já que não há um banco de dados disponível. Além disso, a interface é sobrecarregada com informações mal distribuídas, o que compromete a experiência, especialmente para usuários que buscam algo mais prático e direto. Apesar dessas limitações, o grande diferencial do aplicativo está na seção de treinos: é possível acessar fichas de treino utilizadas por atletas patrocinados pela marca, o que pode ser um atrativo importante para praticantes de musculação que buscam inspiração ou orientação baseada em referências reais do meio esportivo.

Por fim, o MyFitnessPal se destaca como uma aplicação bastante completa, especialmente em sua versão premium. Ele oferece uma ampla gama de funcionalidades, com uma interface intuitiva e bem organizada. Um dos seus maiores pontos fortes é o banco de dados extremamente robusto, que inclui tanto alimentos básicos quanto itens cadastrados por outros usuários da comunidade. Essa contribuição coletiva enriquece significativamente o app, tornando o processo de registro de refeições mais rápido e preciso. Embora a versão gratuita apresente suas limitações.

A análise dos trabalhos relacionados e dos aplicativos estudados traz a possibilidade de evidenciar que, embora existam funcionalidades disponíveis para o registro alimentar, todas elas apresentam limitações que impactam diretamente a praticidade e a personalização necessárias para um uso diário, características fundamentais para esse tipo de solução. Enquanto o MyFitnessPal destaca-se pela ampla base de dados, ele restringe funcionalidades importantes — inclusive funcionalidades principais — a uma versão paga, além de limitar a flexibilidade do registro. O Alimente-se oferece uma boa experiência visual e usabilidade sólida, mas possui banco de dados limitado e análises que nem sempre se adaptam ao perfil do usuário. Já o aplicativo da Growth apresenta sérias barreiras de uso, tornando o registro alimentar pouco viável na prática.

Neste cenário, a aplicação BiteUp posiciona-se como uma alternativa mais equilibrada ao unir simplicidade no registro, possibilidade de personalização para cada usuário, uso de \textit{feedback} subjetivo para uma análise individualizada e um menu principal focado na simplicidade e usabilidade, destacando informações verdadeiramente relevantes por meio de uma interface intuitiva e fácil de utilizar. Dessa forma, a proposta busca ocupar um espaço intermediário entre robustez e usabilidade, oferecendo aquilo que os principais concorrentes prometem mas não conseguem entregar: uma aplicação de fácil uso diário, porém completa, por meio de praticidade real com personalização individual acessível.





% COMMENT
\chapter{Solução proposta}
\label{ch:solucao_proposta}

De modo a atender aos objetivos apresentados no Capítulo~\ref{ch:introducao}, fundamentando-se no referencial teórico do Capítulo~\ref{ch:referencial_teorico} e nas contribuições dos trabalhos relacionados do Capítulo~\ref{ch:trabalhos_relacionados}, propõe-se o desenvolvimento da solução Web denominada Track Forms.

A proposta caracteriza-se pela adoção de uma abordagem flexível e controlada, viabilizada pela criação de formulários com estruturas personalizadas para a coleta de dados, com controle de acesso à aplicação e rastreabilidade das informações, por meio da implementação de mecanismos de autenticação e de funcionalidade de trilha de auditoria. A flexibilidade da estrutura dos formulários possibilita a adaptação da ferramenta a diferentes contextos operacionais, favorecendo a coleta centralizada de informações e assegurando a integridade dos dados em conformidade com as exigências dos órgãos reguladores do setor farmacêutico. Entretanto, em ambientes com conectividade limitada, a ausência de recursos adequados pode inviabilizar o uso da ferramenta; por isso, prevê-se a implementação de funcionalidades para operação \textit{offline}, garantindo que o sistema permaneça útil e funcional mesmo em situações de restrição de acesso à internet.

Embora inicialmente voltada para atender às necessidades específicas deste setor, a solução poderá ser aplicada em diversas áreas, especialmente naquelas que exigem armazenamento de dados em ambiente controlado, segurança quanto à integridade das informações e rastreabilidade das ações realizadas no sistema.

\section{Metodologia de Desenvolvimento}

A metodologia adotada para o desenvolvimento da solução fundamenta-se no modelo cascata, que se caracteriza por uma abordagem linear e estruturada, com fases bem definidas para o desenvolvimento e manutenção do software, conforme ilustrado na Figura \ref{fig:4.1}. Essa estrutura apresenta forte similaridade com o ciclo de vida de sistemas computadorizados, conforme descrito no Guia nº 33 \cite{guia33ANVISA2025} e representado na Figura \ref{fig:2.1}, reforçando a aderência da metodologia às boas práticas preconizadas para setores regulados, como o caso da indústria farmacêutica.

\begin{figure}[!ht]
        \centerline{\includegraphics[width=33em]{capitulo4/img/modeloCascata.png}}
        \caption{Modelo Cascata}
        \label{fig:4.1}
        \centerline{{Fonte: \cite{sommerville2011engenharia}}}
\end{figure}

Segundo \cite{sommerville2011engenharia}, o modelo em cascata considera as atividades fundamentais do processo de especificação, desenvolvimento, validação e evolução, e representa cada uma delas como fases distintas, como: especificação de requisitos, projeto de software, implementação, teste e assim por diante. Trata-se de uma abordagem sequencial que promove o desenvolvimento estruturado, no qual cada fase deve ser concluída antes do início da seguinte. Essa característica facilita a formalização da documentação correspondente a cada etapa do projeto e contribui para o acompanhamento do progresso do desenvolvimento.

\subsection{Definições de Requisitos}

Considerando que o objetivo principal deste projeto é disponibilizar uma solução para a coleta e o armazenamento de dados heterogêneos, com foco em integridade dos dados conforme ALCOA++, e que ele não está sendo desenvolvido para fins comerciais ou para uma aplicação específica em ambiente corporativo, os requisitos mínimos do sistema foram definidos com base nos princípios de integridade de dados, além de contemplarem aspectos de usabilidade, segurança dos dados da aplicação e controles específicos para assegurar o atendimento regulatório. A definição desses requisitos baseou-se em referências normativas vigentes, conforme apresentado nas Tabelas \ref{tab:tabela_a1} e \ref{tab:tabela_a2} presentes no Apêndice \ref{append:requistos}.

\subsection{Projeto do Sistema/ Software}

Após a definição dos requisitos do sistema, foi realizada a seleção das tecnologias a serem adotadas no desenvolvimento do projeto. Essa escolha considerou os requisitos definidos para a aplicação, as boas práticas de desenvolvimento de software e o domínio técnico do desenvolvedor. O objetivo foi assegurar a entrega de uma solução de qualidade, alinhada aos objetivos estabelecidos para o projeto.

\subsubsection{Arquitetura do Sistema}

% A arquitetura do sistema foi definida com base no modelo de comunicação \textit{RESTful}. Essa decisão fundamentou-se no requisito funcional que envolvem a necessidade de permitir o funcionamento da aplicação em modo offline, além de viabilizar a implementação futura de interfaces com outros sistemas. A utilização da \textit{API RESTful} permitirá que os dados sejam disponibilizados de forma estruturada e acessível, facilitando o consumo da API, a integração com os recursos adotados para o funcionamento offline e a sincronização dos dados. Além disso, essa abordagem favorece a compatibilidade com diferentes interfaces, como navegadores e aplicações móveis.
A arquitetura do sistema foi definida com base no modelo de comunicação \textit{RESTful}, cuja abordagem facilita a interoperabilidade e a compatibilidade com diferentes interfaces, como navegadores e aplicativos móveis, além de favorecer futuras integrações com outros sistemas. A utilização da \textit{API RESTful} permite organizar os dados em formato padronizado, como JSON, o que simplifica o consumo das informações e torna mais eficiente a sincronização entre cliente-servidor.

\subsubsection{Camada de Back-end}

Para o desenvolvimento da camada de \textit{back-end}, optou-se pela linguagem de programação Java, amplamente utilizada pela comunidade de desenvolvedores e reconhecida por sua abordagem orientada a objetos. A utilização de programação orientada a objetos favorece a organização da lógica da aplicação e a definição estruturada dos modelos de dados por meio de classes, possibilitando modularidade, reutilização de código e maior eficiência no processo de desenvolvimento e manutenção do software.

Além disso, foi adotado o uso do \textit{framework Spring Boot}, com o objetivo de otimizar o desenvolvimento do projeto, aproveitando sua ampla gama de dependências e configurações automatizadas. Essa escolha visou facilitar a implementação de funcionalidades essenciais, como autenticação, integração com o banco de dados e criação de APIs.

\subsubsection{Banco de Dados}
\label{sec:banco-de-dados}

No que tange à persistência de dados, foi adotado um banco de dados não relacional, o MongoDB, devido à sua capacidade de armazenar informações no formato de documentos. Essa abordagem oferece maior liberdade para modelar os diferentes tipos de formulários que possam vir a ser criados na aplicação, permitindo que cada um contenha uma estrutura específica de acordo com as necessidades do processo. 

A escolha mostrou-se especialmente adequada diante da natureza dinâmica da aplicação, sobretudo no que se refere à variação na quantidade de perguntas que cada formulário venha a ter. A flexibilidade exigida por essa característica, ou seja, a possibilidade de os formulários apresentarem estruturas variáveis e não padronizadas, dificultaria o uso de bancos de dados relacionais, baseados em esquemas fixos e tabelas previamente estruturadas, que imporiam limitações significativas à implementação da solução proposta.

\subsubsection{Camada de Front-end}

A camada de \textit{front-end} será estruturada como um módulo independente, desacoplado da camada de \textit{back-end}, em conformidade com a arquitetura baseada em \textit{API RESTful}.

Para o desenvolvimento das interfaces, estão sendo consideradas abordagens que favoreçam a reutilização de componentes visuais e a construção de elementos dinâmicos, em razão da estrutura não padronizada dos formulários quanto ao número de perguntas. Entre as tecnologias avaliadas, destaca-se a biblioteca \textit{React JS}, pela sua capacidade de lidar com estados e componentes interativos. Como alternativa, também é considerada a utilização de \textit{JavaScript} puro, com manipulação direta do \textit{DOM}, em cenários de menor complexidade.

\subsubsection{Funcionalidade Offline com PWA}

Visando possibilitar o funcionamento offline da aplicação, será incorporada a abordagem de \textit{Progressive Web App (PWA)}. A implementação desse recurso tecnológico tem como objetivo garantir a acessibilidade em ambientes sem conexão à rede, permitindo o uso da aplicação mesmo em condições de conectividade limitada, como ocorre em grandes plantas industriais.

\subsubsection{Modelagem do Sistema}

Com base nos requisitos funcionais e no fluxo de gestão de documentações, elaborado em referência ao Capítulo V da \cite{ANVISARDC658}, foi realizada a modelagem do sistema. Esses documentos encontram-se apresentados nos Apêndices \ref{tab:tabela_a2} e \ref{fig:diagramProcesso}

A modelagem está representada graficamente por meio de um diagrama UML, apresentado no Apêndice \ref{fig:diagramaUML}, o qual evidencia a sequência lógicas das atividades, bem como as interações entre os atores e as funcionalidades previstas.

Para representar a estrutura interna do sistema, foi elaborado o diagrama de classes, apresentado no Apêndice \ref{fig:diagramaClasse}, o qual demonstra as principais classes da aplicação, seus respectivos atributos, métodos e os relacionamentos entre elas. Este modelo orientado a objetos foi construído com base nos casos de uso e nos requisitos do sistema, com o objetivo de assegurar uma implementação coerente e adequada da aplicação.

\subsection{Implementação}

Nesta seção são apresentadas as etapas que compõem a implementação da solução proposta. O processo de desenvolvimento foi conduzido de maneira estruturada, iniciando pela criação da base do projeto a partir de um \textit{framework de back-end}, passando pela construção dos modelos e concluindo com o desenvolvimento das funcionalidades da aplicação.

\subsubsection{Projeto Spring Boot}

A implementação do sistema teve início com a criação da base lógica e estrutural da aplicação, estabelecida na camada de back-end. Utilizou-se a ferramenta web \textit{Spring Initializr} \footnote{Disponível em: https://start.spring.io/} para gerar o projeto com base no \textit{framework Spring Boot}. Nessa etapa, foram definidos os parâmetros mínimos e essenciais para a criação da estrutura básica do projeto, como o sistema de \textit{build}, a linguagem de programação e a versão do \textit{framework} adotado, além da inclusão das dependências desejadas. Optou-se pelo sistema de gerenciamento \textit{Maven}, pela linguagem Java (versão 17) e pela versão 3.5.6 do \textit{framework}, conforme demonstrado na Figura \ref{fig:4.2}.

A integração com os principais componentes necessários do ecossistema Spring Boot para o desenvolvimento da solução foi estabelecida por meio da inclusão das dependências: \textit{Spring Security}, \textit{Spring Web} e \textit{Spring Data MongoDB}. Além das dependências principais, foram incluídas dependências auxiliares, como \textit{Spring Boot DevTools}, \textit{Lombok} e \textit{Validation}, com o objetivo de facilitar e otimizar o desenvolvimento da solução.

\begin{itemize}
  \item \textbf{Spring Security}: utilizada para a implementação de autenticação e controle de acesso;
  \item \textbf{Spring Web}: utilizada para a estruturação da camada de controle da aplicação, por meio de APIs RESTful e do tratamento de requisições HTTP;
  \item \textbf{Spring Data MongoDB}: utilizada para a comunicação e persistência dos dados no banco de dados não relacional MongoDB.
  \item \textbf{Spring Boot DevTools}: utilizada para possibilitar atualizações automáticas durante o desenvolvimento e a execução do projeto.
  \item \textbf{Lombok}: utilizada para reduzir a verbosidade do código, evitando a criação manual de construtores e métodos padrão, como \textit{getters} e \textit{setters}.
  \item \textbf{Validation}: utilizada para validar os dados fornecidos pelos usuários antes do processamento das requisições na API.
\end{itemize}

\begin{figure}[!h]
        \centerline{\includegraphics[width=30em]
        {capitulo4/img/frameWork.png}}
        \caption{Configuração do projeto definida no Spring Initializr}
        \label{fig:4.2}
        \centerline{Fonte: Autoria própria}
\end{figure}

Com o objetivo de auxiliar no desenvolvimento da aplicação, principalmente no controle de alterações, e de modo a mantê-la como parte do portfólio do desenvolvedor, foi criado um repositório privado no GitHub: \url{https://github.com/rodr1golm/TrackForms.git}. Esse repositório serve como base para a gestão e acompanhamento do progresso do projeto, além de permitir o controle das alterações realizadas.

A decisão de manter o repositório privado foi motivada pelo fato de que, embora a aplicação não tenha sido concebida com fins comerciais, a proposta do projeto apresenta potencial relevante na área farmacêutica, podendo ser aprimorada futuramente para uso comercial.

\subsubsection{Desenvolvimento}

O desenvolvimento do projeto foi realizado utilizando o \textit{Visual Studio Code (VS Code)} como ambiente de desenvolvimento, em conjunto com a extensão \textit{Spring Boot Dashboard}, responsável por facilitar a execução do projeto e a inicialização do servidor interno do Spring Boot de maneira simplificada. Além disso, foi utilizado o \textit{Postman} para a execução de testes das requisições enviadas à API, possibilitando a validação dos endpoints e o monitoramento das respostas.

\subsubsection{Estrutura do Projeto}

A estrutura interna do projeto na camada de \textit{back-end} foi desenvolvida com base na arquitetura \textit{MVC (Model-View-Controller)}, adaptada ao contexto de uma \textit{API RESTful}, com o objetivo de promover a separação de responsabilidades entre as camadas da aplicação. Essa abordagem permite que a lógica de negócio, a interface de usuário e o controle de fluxo sejam desenvolvidos de forma independente, resultando em um código mais organizado, modular e de fácil manutenção.

A camada \textit{Model} é responsável por representar as entidades do domínio da aplicação. Ela é composta pelas classes principais \textit{Formulario}, \textit{Resposta}, \textit{TrilhaAuditoria} e \textit{Usuario}, que definem a estrutura central dos objetos da aplicação. Complementarmente, ela inclui as classes auxiliares \textit{Pergunta} e \textit{Historico}, utilizadas na composição da entidade Formulário.

As classes \textit{Formulario}, \textit{Resposta} e \textit{Usuario} incorporam campos de metadados que contribuem diretamente para a rastreabilidade das alterações, em conformidade com o atributo Rastreável, conforme definido pelo princípio ALCOA++. Um exemplo dessa estrutura pode ser verificado na classe \textit{Formulario}, conforme analisado no Código-fonte \ref{codigo-fonte:modeloFormulario}, que demonstra a implementação dos metadados na classe que representa o objeto de domínio.

\begin{listing}[!h]
\jscode{capitulo4/codigos/Formulario.java}
\caption{Objeto de domínio da classe Formulário}
\label{codigo-fonte:modeloFormulario}
\end{listing}

Entre os componentes desenvolvidos, a classe \textit{Formulario} tem grande importância na aplicação, pois ela é a responsável por viabilizar a criação de formulários com estruturas heterogêneas, por meio da composição com a classe Pergunta. Essa composição permite a definição de uma lista de perguntas, criadas conforme as necessidades específicas do usuário, e que serão persistidas no banco de dados junto ao respectivo formulário.

A classe \textit{TrilhaAuditoria} também tem grande relevância, pois ela é a encarregada de estruturar o modelo utilizado para o registro das ações realizadas na aplicação, conforme apresentado no Código-fonte \ref{codigo-fonte:modeloTrilhaAuditoria}. Essa estrutura integra a implementação da funcionalidade de trilha de auditoria, composta pelo histórico de atividades relevantes, como cadastros, alterações de formulários, submissões de respostas e eventos críticos. Ao implementar essa funcionalidade, o sistema passa a atender plenamente ao atributo de Rastreabilidade, garantindo a disponibilidade das informações sobre as ações executadas pelos usuários. Essa implementação possibilita o registro e o acompanhamento de eventos relevantes, em conformidade com o Requisito Funcional item 11 da tabela \ref{tab:tabela_a2}, presente no Apêndice \ref{append:requistos}, o qual integra os requisitos de usuários que fundamentam o desenvolvimento do projeto.

\begin{listing}[!h]
\jscode{capitulo4/codigos/TrilhaAuditoria.java}
\caption{Objeto de domínio da classe Trilha de Auditoria}
\label{codigo-fonte:modeloTrilhaAuditoria}
\end{listing}

Considerando a relevância do requisito relacionado ao atributo Rastreabilidade, será utilizado a funcionalidade da Trilha de Auditoria para exemplificar a estrutura das demais camadas da aplicação.

A camada \textit{View} é responsável pela interface de usuário, atuando como ponte entre o processo de solicitação e a renderização das informações na aplicação. Em uma arquitetura baseada em \textit{API RESTful}, essa camada não está presente no \textit{back-end}, mas sim no \textit{front-end}, que consome os dados fornecidos pela API e os apresenta ao usuário por meio de componentes visuais e interativos. Essa camada será desenvolvida em uma etapa posterior, durante a segunda fase do desenvolvimento do Trabalho de Conclusão de Curso.

A camada \textit{Controller} é responsável por intermediar a comunicação entre o cliente e as demais camadas da aplicação, processando as requisições recebidas e coordenando as respostas apropriadas. Como exemplificado pela classe \textit{TrilhaAuditoriaController} no código-fonte \ref{codigo-fonte:TrilhaAuditoriaController}, o Controller não se comunica diretamente com a View, mas sim com a camada \textit{Service}. Esta, por sua vez, centraliza a lógica de negócio da aplicação, atuando como intermediária entre  os \textit{Controllers} e os \textit{Repositories}.

\begin{listing}[!h]
\jscode{capitulo4/codigos/TrilhaAuditoriaController.java}
\caption{Controller da classe Trilha de Auditoria}
\label{codigo-fonte:TrilhaAuditoriaController}
\end{listing}

Conforme o exemplo demonstrado na classe \textit{TrilhaAuditoriaService}, código-fonte \ref{codigo-fonte:TrilhaAuditoriaService}, essa classe é a responsável por criar e registrar os logs de trilha de auditoria por meio do método \textit{Registrar}, utilizando a \textit{TrilhaAuditoriaFactory} para a construção dos objetos e o \textit{TrilhaAuditoriaRepository} para a persistência dos dados. Além disso, é responsável por realizar a busca dos registros por meio do método \textit{ListarTrilhas}, que retornará os logs serializados em formato \textit{JSON}, posteriormente consumidos pela interface de usuário no \textit{front-end}.

O método \textit{Registrar} desempenha um papel central na implementação, sendo reutilizado em todos os eventos de alteração realizados por meio dos métodos \textit{POST}, \textit{PUT} e \textit{DELETE}, bem como em consultas críticas via \textit{GET}. Seu objetivo é registrar no histórico as atividades relevantes, incluindo cadastros, modificações de formulários, submissões de respostas e buscas de informações de acesso restrito.

\begin{listing}[!h]
\jscode{capitulo4/codigos/TrilhaAuditoriaService.java}
\caption{Service Trilha de Auditoria}
\label{codigo-fonte:TrilhaAuditoriaService}
\end{listing}

E por fim, a camada \textit{Repository} é responsável pela lógica de acesso aos dados, incluindo a persistência e recuperação de informações no banco de dados não relacional \textit{MongoDB}. Essa funcionalidade é viabilizada por meio da dependência \textit{Spring Data MongoDB}, integrada ao \textit{framework} de \textit{back-end}, que fornece abstrações e interfaces para simplificar operações de leitura, escrita e consulta dos dados.

\chapter{Resultados Parciais}
\label{ch:resultados_discussoes}

Com base no desenvolvimento parcial da solução, é possível observar o cumprimento de alguns dos objetivos previamente estabelecidos. Entre eles, destacam-se a implementação de um meio centralizado para a coleta e o armazenamento de informações heterogêneas, comprovada pelas funcionalidades já desenvolvidas relacionadas à obtenção e à persistência dos dados, e o atendimento aos requisitos de integridade, considerado um dos objetivos mais relevantes do projeto, dada a sua importância na indústria farmacêutica, onde a confiabilidade das informações é essencial para garantir segurança, rastreabilidade e conformidade regulatória.

\section{Coleta de Informações Heterogêneas}

Por meio das funcionalidades para cadastro de formulários e para submissão das respectivas respostas, evidencia-se a possibilidade de, em uma única aplicação, trabalhar com diferentes estruturas de dados de forma centralizada, permitindo a coleta de informações heterogêneas de maneira organizada e consistente. 

Essa característica pode ser observada na Figura \ref{fig:5.1}, que apresenta a lista de formulários cadastrados com estruturas distintas em formato \textit{JSON}, exibidas por meio da função \textit{preview} do \textit{Postman}, de modo a facilitar a visualização do exemplo. Neste caso, existem três formulários cadastrados: um deles contendo apenas uma pergunta de verdadeiro ou falso (dado do tipo \textit{boolean}); outro composto por oito perguntas diversas, com diferentes tipos de conteúdo de resposta; e um terceiro com três perguntas voltadas para a medição de temperatura (dado do tipo \textit{float}).

\begin{figure}[!h]
        \centerline{\includegraphics[width=35em]
        {capitulo5/img/formularios.png}}
        \caption{Consulta de formulários cadastrados na aplicação.}
        \label{fig:5.1}
        \centerline{Fonte: Autoria própria}
\end{figure}

As Figuras \ref{fig:5.2}, \ref{fig:5.3} e \ref{fig:5.4} apresentam a lista de respostas dos formulários cadastrados, permitindo observar as variações na composição dos objetos \textit{JSON} conforme o conteúdo de cada formulário. Essas variações estão diretamente relacionadas ao tipo de dado esperado para cada resposta e à quantidade de perguntas do formulário, refletindo as particularidades da estrutura definida em cada um deles.

\begin{figure}[!h]
        \centerline{\includegraphics[width=35em]
        {capitulo5/img/respostasForm1.png}}
        \caption{Exemplo de resposta ao formulário de Pesquisa de opinião.}
        \label{fig:5.2}
        \centerline{Fonte: Autoria própria}
\end{figure}

\begin{figure}[!h]
        \centerline{\includegraphics[width=35em]
        {capitulo5/img/respostasForm2.png}}
        \caption{Exemplo de resposta ao formulário de \textit{checklist} de monitoramento.}
        \label{fig:5.3}
        \centerline{Fonte: Autoria própria}
\end{figure}

\begin{figure}[!h]
        \centerline{\includegraphics[width=35em]
        {capitulo5/img/respostasForm3.png}}
        \caption{Exemplo de resposta ao formulário de Monitoramento de Temperatura.}
        \label{fig:5.4}
        \centerline{Fonte: Autoria própria}
\end{figure}

\section{Persistência dos Dados}
\label{sec:persistenciaDados}

A persistência dos dados está funcional e foi implementada por meio do banco de dados \textit{MongoDB}, conforme descrito Seção \ref{sec:banco-de-dados}. As informações relacionadas aos formulários são organizadas em três coleções distintas, conforme apresentado na Figura \ref{fig:5.5}, que exibe a visualização da estrutura do banco de dados \textit{TrackForms} por meio do \textit{MongoDB Compass}, interface gráfica utilizada para o acesso e a inspeção dos dados armazenados.

\begin{itemize}
  \item A coleção \textit{formularios} é responsável por armazenar a estrutura dos formulários criados, bem como seus metadados;
  \item A coleção \textit{respostas} é utilizada para registrar as respostas submetidas pelos usuários. Para cada submissão, é criado um objeto contendo as respostas individuais, acompanhadas de metadados que as correlacionam ao respectivo formulário;
  \item A coleção \textit{logs\_audit\_trail} é destinada ao armazenamento de eventos e ações realizadas no sistema, operando como trilha de auditoria da aplicação.
\end{itemize}

\begin{figure}[!h]
        \centerline{\includegraphics[width=35em]
        {capitulo5/img/mongodb.png}}
        \caption{Base de dados TrackForms no MongoDB Compass}
        \label{fig:5.5}
        \centerline{Fonte: Autoria própria}
\end{figure}

\section{Integridade dos Dados}

Em conformidade com os princípios de integridade de dados e alinhada aos atributos estabelecidos pelo modelo ALCOA++, a aplicação foi projetada e implementada com o propósito de atender aos requisitos, conforme estabelecido na Tabela \ref{tab:table_3_1}.

Conforme ilustrado nas Figuras \ref{fig:5.2}, \ref{fig:5.3} e \ref{fig:5.4}, os objetos criados na aplicação incorporam os campos específicos “criado por” e “modificado por”, que permitem identificar o responsável pela criação e alteração dos dados, vinculando-os ao usuário autenticado no momento da ação e atendendo ao atributo \textbf{Rastreável}.
Além desses campos, os objetos também incluem metadados e carimbos de data e hora, que viabilizam a rastreabilidade temporal das operações realizadas. Dessa forma, a aplicação atende ao atributo \textbf{Completo}, por meio da implementação de registros abrangentes que contemplam tanto os dados principais quanto os elementos complementares.

O atributo \textbf{Legível} é atendido pela implementação da solução informatizada, que, por meio da persistência dos registros digitais, viabiliza a digitalização dos processos. Essa abordagem favorece a interpretação das informações, promovendo maior confiabilidade na análise dos dados e contribuindo para a mitigação de erros recorrentes em registros manuscritos, como rasuras e caligrafia ilegível

O atributo \textbf{Original} é atendido por meio do registro dos dados em ambiente controlado, representado pelo banco de dados, garantindo a preservação e a segurança das respostas submetidas. Essa abordagem assegura que não ocorram alterações posteriores à submissão, mantendo a autenticidade e a confiabilidade das informações registradas.

O atributo \textbf{Consistente} é assegurado por meio do fluxo estruturado para gestão dos formulários, estabelecido para controlar tanto a criação quanto a revisão dos formulários, além do controle de submissão de respostas. 

Sob a perspectiva deste processo, estabeleceram-se regras que garantem a utilização apenas de documentos válidos, a rastreabilidade do histórico de revisões e a manutenção de todas as versões existentes dos formulários.

Conforme apresentado na Figura \ref{fig:5.6}, cada formulário é inicialmente criado com o status de \textit{Elaboração} e deve passar por aprovação de um usuário de nível Gestor. O status do documento pode ser alterado para \textit{Ativo} se a aprovação for concedida, o que está condicionado à existência de perguntas no formulário, ou ser reprovado, tendo seu status alterado para \textit{Cancelado}. A utilização para coleta de dados somente pode ser realizada se o formulário estiver com status \textit{Ativo}. Quando houver necessidade de revisão, o usuário Gestor poderá criar uma nova versão do formulário, a qual incorpora automaticamente o histórico do documento que a originou. O formulário revisado tem seu status alterado para \textit{Obsoleto}, bloqueando seu uso e garantindo que apenas documentos válidos e atualizados sejam empregados no processo. Esses mecanismos foram concebidos com base em uma lógica voltada para preservar a originalidade das informações, mitigar potenciais falhas operacionais e assegurar o funcionamento adequado da aplicação.

\begin{figure}[!h]
        \centerline{\includegraphics[width=30em]
        {capitulo5/img/fluxoStatus.png}}
        \caption{Fluxograma de status do Formulário}
        \label{fig:5.6}
        \centerline{Fonte: Autoria própria}
\end{figure}

O atributo \textbf{Rastreável} é plenamente atendido por meio da funcionalidade de trilha de auditoria implementada na aplicação, a qual permite o monitoramento detalhado de todas as ações críticas realizadas, incluindo o cadastro, as alterações nos formulários e as submissões de respostas. Essa funcionalidade promove a rastreabilidade e a transparência das ações executadas no sistema, conforme ilustrado na Figura \ref{fig:5.7}, contribuindo para a rastreabilidade das ações e a garantia da integridade dos dados.

\begin{figure}[!h]
        \centerline{\includegraphics[width=35em]
        {capitulo5/img/trilhaAuditoria.png}}
        \caption{Registros da Trilha de Auditoria}
        \label{fig:5.7}
        \centerline{Fonte: Autoria própria}
\end{figure}

A conformidade com o atributo \textbf{Duradouro} e \textbf{Disponível} é assegurada pela persistência dos dados em um banco de dados não relacional, conforme demonstrado na seção \ref{sec:persistenciaDados}. No entanto, destaca-se a necessidade de implementação de mecanismos complementares de \textit{backup} e recuperação, a fim de garantir que os dados permaneçam acessíveis e íntegros durante todo o ciclo de vida da informação.

Por fim, o atributo \textbf{Contemporâneo} será contemplado em uma etapa posterior, por meio da implementação de mecanismos de controle temporal para a submissão dos formulários. Essa funcionalidade garantirá que as informações sejam registradas no banco de dados dentro de um intervalo de tempo definido após o preenchimento das informações na aplicação, assegurando a temporalidade e a conformidade com os requisitos regulatórios.




\begin{singlespace}
\begin{flushleft}
% referencias
% estilo de bibliografia no formato ABNT
\bibliographystyle{abnt-ufrgs}

% aqui chama as referências descritas no formato .bib
% espeficamente é chamado o arquivo references.bib
% observe que os arquivos .bib tem formato específico e basta
% descrever as referências nesse formato e tudo será gerado automaticamente
\bibliography{references}
\end{flushleft}
\end{singlespace}



% apendices

\renewcommand{\appendixname}{AP\^ENDICE~}
\begin{appendices}
	\chapter[\hspace{1mm}]{- Diagramas de Caso de Uso} 
\label{append:published_papers}

{\large \textbf{Diagramas}}

Adicionar aqui os diagramas ...
	\chapter[\hspace{1mm}]{- Diagramas} 
\label{append:published_papers}

{\large \textbf{Diagramas}}

\begin{figure}[!h]
        \centerline{\includegraphics[width=40em]
        {apendices/img/fluxogramaProcesso.png}}
        \caption{Fluxograma de Processo}
        \label{fig:diagramProcesso}
        \centerline{Fonte: Autoria própria}
\end{figure}

\begin{figure}[!h]
        \centerline{\includegraphics[width=40em]
        {apendices/img/casosUso.png}}
        \caption{Diagrama UML de Casos de Uso}
        \label{fig:diagramaUML}
        \centerline{Fonte: Autoria própria}
\end{figure}

\begin{figure}[!h]
        \centerline{\includegraphics[width=40em]
        {apendices/img/diagramaClasse.png}}
        \caption{Diagrama de Classes}
        \label{fig:diagramaClasse}
        \centerline{Fonte: Autoria própria}
\end{figure}

\end{appendices}

\end{document}
