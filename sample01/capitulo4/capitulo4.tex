\chapter{Solução proposta}
\label{ch:solucao_proposta}

Com o objetivo de facilitar a consulta de macronutrientes por parte de usuários, sejam atletas ou indivíduos que buscam manter uma alimentação saudável, e também de otimizar a prescrição dietética por profissionais da área de nutrição, este projeto propõe o desenvolvimento de uma plataforma capaz de fornecer informações nutricionais detalhadas de cada alimento. A partir dos dados fornecidos pelos usuários, a aplicação calcula automaticamente metas nutricionais personalizadas, alinhadas aos objetivos individuais definidos por cada usuário, tais como ganho de massa muscular ou perda de gordura corporal.

A plataforma oferece uma interface prática e intuitiva, voltada para o uso diário, permitindo o acompanhamento em tempo real do impacto de cada alimento na meta nutricional estabelecida. Dessa forma, o usuário pode visualizar como suas escolhas alimentares contribuem positiva ou negativamente para o alcance de seus objetivos.

Com base em três fontes de dados: os alimentos consumidos informados pelo próprio usuário, um banco de dados nutricional pré-definido e alimentos cadastrados manualmente pelo usuário, o sistema calcula os macronutrientes principais: proteínas, carboidratos e gorduras totais. Esses valores são atualizados em tempo real na interface do usuário, permitindo uma visão clara do progresso diário em relação às metas.

Um dos principais diferenciais do projeto reside na implementação de uma funcionalidade baseada em aprendizado de máquina. A aplicação coleta, analisa e interpreta os hábitos alimentares do usuário ao longo do tempo, aprendendo continuamente com os dados registrados. Com isso, torna-se capaz de gerar recomendações personalizadas e inteligentes, que consideram tanto os objetivos definidos quanto o histórico de alimentação e o bem-estar relatado pelo próprio usuário.

Essa funcionalidade não apenas sugere ajustes pontuais na dieta, mas também propõe versões alternativas de refeições, combinações de alimentos mais adequadas e até planos alimentares completos adaptados a restrições específicas, tais como intolerâncias, alergias ou preferências alimentares. Com essa abordagem dinâmica e personalizada, o sistema visa promover uma experiência única para cada usuário, elevando a qualidade do acompanhamento nutricional e a aderência às metas traçadas.

\section{Tecnologias utilizadas}

Para a implementação da plataforma proposta, foram escolhidas tecnologias amplamente consolidadas no mercado e com boa curva de aprendizado, visando facilitar o desenvolvimento, os testes e a futura manutenção da aplicação. A arquitetura geral adotada é baseada no modelo cliente-servidor, com uma clara separação entre \textit{front-end}, \textit{back-end} e banco de dados.

\\textbf{\\textit{Back-end}:} Desenvolvido com \\textit{Node.js}\\footnote{\\url{https://nodejs.org}} e \\textit{Express}\\footnote{\\url{https://expressjs.com}}. A escolha dessa \\textit{stack} se deu por sua leveza, grande ecossistema de bibliotecas e facilidade de construção de \\textit{APIs RESTful}. O \\textit{Express} permite criar rotas, \\textit{middlewares} e controladores de forma organizada, possibilitando uma manutenção mais simples do código.

Os modelos e controladores da API já estão funcionando para as entidades existentes, com rotas \textit{REST} implementadas e testadas com sucesso através da ferramenta \textit{Insomnia}. As seguintes operações estão disponíveis, para cada entidade presente no projeto:
    \begin{itemize}
        \item GET / lista todos elementos da entidade;
        \item GET / :id busca elemento da entidade através do id;
        \item POST / – cria um novo elemento com os dados informados.
        \item PUT /:id – atualiza os dados do elemento;
        \item DELETE /:id – exclui um determinado elemento através do Id.
    \end{itemize}
    


\textbf{Banco de Dados:} Utiliza \textit{MySQL}\footnote{\url{https://www.mysql.com}} para armazenamento relacional. A modelagem do banco segue um esquema estruturado com tabelas relacionadas, garantindo integridade referencial e eficiência em consultas. O \textit{XAMPP}\footnote{\url{https://www.apachefriends.org}} está sendo utilizado como ambiente local para facilitar o gerenciamento do servidor \textit{MySQL} durante o desenvolvimento.

A estrutura do banco de dados relacional foi implementada conforme o modelo lógico previamente elaborado (Figura~\ref{fig:er_diagrama}), com tabelas normalizadas que representam as entidades essenciais do sistema: usuários, alimentos, refeições, \textit{feedbacks} e registros diários. 

\begin{figure}[H]
    \centering
    \includegraphics[width=0.85\textwidth]{capitulo4/img/er1.png}
    \caption{Modelo entidade-relacionamento do banco de dados da aplicação}
    \label{fig:er_diagrama}
\end{figure}

O relacionamento entre as entidades permite associar alimentos a refeições por meio da tabela intermediária \textit{ComposicaoRefeicao}, que registra a quantidade de cada alimento em uma refeição. As refeições estão vinculadas a dias específicos, e todos os registros são associados a um usuário, permitindo um controle individualizado. Além disso, a resposta fornecida pelo usuário é relacionada a uma refeição específica, possibilitando a coleta de dados sobre a experiência alimentar. Essas informações são fundamentais para o funcionamento do sistema de recomendação inteligente, que utilizará esses dados para sugerir melhorias nas escolhas alimentares do usuário.



\\textbf{\\textit{Front-end}:} Desenvolvido com \\textit{Vue.js}\\footnote{\\url{https://vuejs.org}}, um \\textit{framework JavaScript} progressivo que oferece uma abordagem reativa e modular para construção de interfaces dinâmicas. \\textit{Vue} foi escolhido pela sua curva de aprendizado amigável, forte comunidade e capacidade de criar componentes reutilizáveis.

O protótipo da aplicação desenvolvido faz uso do conceito de \textit{mobile first}, onde o foco inicial da arquitetura e desenvolvimento é direcionado aos dispositivos móveis, a tela de protótipo apresenta uma implementação inicial da ideia de consumos, onde o usuário é capaz de visualizar os macronutrientes diários e registrar alimentos.

\begin{figure}[H]
    \centering
    \includegraphics[width=0.5\textwidth]{capitulo4/img/front.png}
    \caption{protótipo front-end}
    \label{fig:front}
\end{figure}

O botão "Bite+" presente no \textit{navbar} executa um evento \textit{emit} que é ouvido pelo \textit{App.vue} e através da diretiva \textit{v-if} exibe o componente com os alimentos disponíveis no banco de dados. Ao selecionar um alimento, seus valores nutricionais são automaticamente somados aos totais exibidos na parte superior da interface, que apresenta a ingestão acumulada e as metas diárias de cada macronutriente (proteína, carboidrato e gordura). A Figura \ref{fig:front} ilustra a estrutura atual da interface, destacando a barra de navegação com o botão de ativação do evento, bem como as \textit{views home} e \textit{daily}, já integradas ao sistema de roteamento da aplicação. Este último é gerenciado pelo componente \textit{Vue Router}, responsável pelo controle de rotas e pela navegação fluida entre as diferentes páginas da aplicação.


\textbf{Ferramentas de desenvolvimento:} Para simulação e testes de requisições \textit{REST}, utiliza-se o \textit{Insomnia}\footnote{\url{https://insomnia.rest}}, que permite validar endpoints e testar fluxos de autenticação, registro de dados e consultas. O \textit{XAMPP} também auxilia no gerenciamento local do servidor e do banco de dados.

Essas tecnologias, em conjunto, possibilitam o desenvolvimento de uma aplicação moderna, modular e escalável, alinhada aos objetivos do projeto.

\section{Fluxo do sistema BiteUp}

A aplicação proposta visa gerenciar e processar informações nutricionais de forma personalizada para cada usuário, acompanhando suas metas diárias de macronutrientes com base em dados pessoais e no histórico de consumo. O fluxo de funcionamento do sistema pode ser dividido em algumas etapas principais que se conectam para entregar uma experiência completa e personalizada.

\begin{figure}[H]
    \centering
    \includegraphics[width=0.5\textwidth]{capitulo4/img/cadastrar.png}
    \caption{cadastro}
    \label{fig:cadastro}
\end{figure}

\textbf{Cadastro e perfil do usuário:} No primeiro acesso, o usuário é convidado a criar seu perfil, preenchendo informações essenciais como nome, idade, sexo, altura, peso atual e objetivo (ganho de massa muscular ou perda de peso) conforme \ref{fig:cadastro}.


\begin{figure}[H]
    \centering
    \includegraphics[width=0.5\textwidth]{capitulo4/img/login.png}
    \caption{login}
    \label{fig:login}
\end{figure}

Após o cadastro do perfil de usuário, já é possível efetuar o login utilizando email e senha conforme figura \ref{fig:login}, caso não haja cadastro, há a opção de efetuar o cadastro logo abaixo.


\begin{figure}[H]
    \centering
    \includegraphics[width=0.5\textwidth]{capitulo4/img/menu.png}
    \caption{menu da aplicação}
    \label{fig:menu}
\end{figure}

\textbf{Cálculo automático de metas nutricionais:} Após o cadastro, o sistema utiliza das informações pessoais para calcular as metas diárias de macronutrientes (carboidratos, proteínas e gorduras). Esse cálculo baseia-se em fórmulas nutricionais validadas por estudos acadêmicos (Capítulo \ref{ch:referencial_teorico}), considerando fatores como gasto energético basal \ref{item:GEB}, nível de atividade física e objetivo definido. As metas geradas são salvas e apresentadas ao usuário como seu objetivo diário, na tela menu presente na figura \ref{fig:menu} estes objetivos diários podem ser visualizados no component com título "Macronutrientes" na parte superior do menu.

\begin{figure}[H]
    \centering
    \includegraphics[width=0.5\textwidth]{capitulo4/img/listaAlimentos.png}
    \caption{lista de alimentos}
    \label{fig:lista}
\end{figure}

\textbf{Registro diário de alimentos:} Para acompanhar o progresso em direção às metas, o usuário deve registrar diariamente os alimentos consumidos. Isto pode ser feito selecionando alimentos personalizados inseridos por ele mesmo, ou pré-cadastrados no banco de dados nutricional, com informações já padronizadas sobre porções e macronutrientes, a funcionalidade pode ser visualizada na figura \ref{fig:lista}.

\begin{figure}[H]
    \centering
    \includegraphics[width=0.5\textwidth]{capitulo4/img/bananaSelecionado.png}
    \caption{alimento selecionado}
    \label{fig:selecao}
\end{figure}

Cada alimento registrado é vinculado a uma refeição de forma automática pelo sistema, de acordo com o horário de registro (café da manhã, almoço, jantar ou lanche), possibilitando ao usuário uma interface visualmente satisfatória do consumo ao longo do dia, ao selecionar o alimento 'banana' por exemplo, ocorre a \textbf{atualização em tempo real do consumo diário} e é criado a refeição 'almoço' presente na figura \ref{fig:selecao} juntamente a um botão com ícone de lâmpada, que permite ao usuário registrar um feedback para a refeição. Conforme os alimentos são registrados, o sistema soma automaticamente os macronutrientes consumidos e atualiza em tempo real o progresso em relação à meta diária. O usuário pode acompanhar essa evolução através da barra de progresso e do valor numérico na interface de macro-nutrientes, visualizando claramente quanto ainda falta consumir ou se já excedeu algum macronutriente conforme figura \ref{fig:selecao}.

\begin{figure}[H]
    \centering
    \includegraphics[width=0.5\textwidth]{capitulo4/img/dailyBanana.png}
    \caption{view daily}
    \label{fig:daily}
\end{figure}


\textbf{Visualização e gestão do histórico:} Todos os registros diários são salvos no banco de dados associado ao usuário e acessados através da view Daily presente na figura \ref{fig:daily}, permitindo consultar os alimentos consumidos, acompanhar tendências ou corrigir uma inserção acidental. Esse histórico é essencial para análises futuras e para o funcionamento do sistema de recomendação inteligente.


\textbf{Personalização com aprendizado de máquina:} O diferencial da aplicação está em sua capacidade de aprendizado contínuo. À medida que o usuário registra refeições e fornece \textit{feedback} sobre o seu bem-estar, o sistema armazena estas informações e utiliza algoritmos de aprendizado de máquina para identificar padrões e ajustar recomendações.

Isso permite que o sistema sugira combinações de alimentos mais adequadas, substituições inteligentes para restrições alimentares ou intolerâncias, e até mesmo planos alimentares personalizados com base no histórico e nos objetivos do usuário, tornando o acompanhamento nutricional progressivamente mais individualizado. A funcionalidade pode ser acessada com um clique do usuário no botão 'gerar insight' presente no componente 'Análise inteligente' do menu ilustrado na figura \ref{fig:menu}.

\section{Funcionamento do sistema de recomendação}

O principal diferencial da aplicação está na funcionalidade de recomendações personalizadas baseadas em aprendizado de máquina. Essa parte do sistema visa analisar o histórico alimentar do usuário, seu perfil e o feedback subjetivo para propor ajustes inteligentes. Os dados armazenados pelo sistema são o \textbf{perfil do usuário} (peso, altura, objetivo), \textbf{registros diários de alimentos e refeições} e \textbf{resposta pessoal sobre sensação e bem-estar}.

A ideia central é permitir que, após registrar um alimento consumido e identificar alguma alteração, como se sentir indisposto por exemplo, o usuário poderá interagir com a plataforma entregando um \textit{feedback} negativo a uma determinada refeição.

Estas respostas serão armazenadas e utilizadas como uma fonte de dados para aprendizado sobre quais alimentos e combinações funcionam melhor para aquele usuário. 
Esse histórico de \textit{feedbacks} passa por um filtro de dados no \textit{back-end}, unindo alimentos relacionados a refeições com \textit{feedbacks} negativos, e, posteriormente entregando estes dados em formato de \textit{prompt} para uma API de IA, que é responsável por identificar padrões mais a fundo e recomendar alterações adequadas e substituições inteligentes que consideram as necessidades individuais do usuário.


\textbf{Lógica por trás da aplicação:}

Para a devida implementação da funcionalidade capaz de auxiliar no controle de macronutrientes, foi desenvolvido um fluxograma responsável por descrever o fluxo da aplicação no quesito de desenvolvimento, facilitando também o entendimento do fluxo de utilização por meio do mesmo (Figura \ref{fig:fluxograma de desenvolvimento}) 

\begin{figure}[H]
    \centering
    \includegraphics[width=1.0\textwidth]{capitulo4/img/fluxog.png}
    \caption{fluxograma explicativo do funcionamento inicial do sistema}
    \label{fig:fluxograma de desenvolvimento}
    
\end{figure}

O fluxograma (Figura \ref{fig:fluxograma de desenvolvimento}) acima ilustra o funcionamento do sistema como foi pensado e como foi desenvolvido.
observando-o é perceptível que um determinado alimento A é ligado a uma tabela intermediária chamada composição refeição cujo intuito trata de intermediar a tabela alimentos com a tabela refeição, já que, uma determinada refeição pode conter diversos alimentos com id's distintos ligados a uma mesma refeição. Cada refeição é controlada pelo id do usuário responsável e pelo id do dia que a mesma foi criada.

\textbf{A tabela usuário} contém todas as informações relacionadas a um determinado usuário, considerando peso, altura, sexo, idade e objetivo, dados que posteriormente são utilizados para calcular os macronutrientes necessários para cada usuário atingir seu objetivo através da aplicação.

\textbf{A tabela dia} contém as colunas data e idUsuario, visto para cada usuário é inserido um dia com id diferente.

\textbf{A tabela Resumo Dia} é responsável por pegar no banco de dados as refeições vinculadas a um determinado dia e somar todos os macronutrientes dos alimentos ligados a Composição Refeição com aquele id de refeição.

\textbf{A tabela \textit{feedback}} possui as colunas descrição e sensação, permitindo ao usuário informar como se sente em relação a uma determinada refeição (informações utilizadas posteriormente para calcular a quantidade de alimentos ligados a uma refeição com \textit{feedback} negativo ou positivo) e mapear quais são responsáveis por mal-estar, ou bem-estar do usuário.

 
 

 

 

