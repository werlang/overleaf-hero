\chapter{Conclusões e perspectivas futuras}
\label{ch:conclusoes_trabalhos_futuso}

Com o intuito de facilitar a consulta e o acompanhamento de macronutrientes ao longo do dia, seja para atletas, usuários comuns ou profissionais da área do esporte, desenvolveu-se uma solução eficaz capaz de promover o acompanhamento nutricional individualizado, considerando critérios de eficiência, agilidade e funcionalidade. A primeira parte deste capítulo apresenta uma análise dos objetivos e resultados que se fez necessário alcançar para tornar a funcionalidade adequada. A segunda parte, respectivamente a parte final, tem o propósito de listar e descrever funcionalidades futuras identificadas ao longo do desenvolvimento deste projeto. Também se faz necessário listar adequações que foram necessárias durante o desenvolvimento e que não haviam sido previstas na versão inicial, mas que se mostraram imprescindíveis para a entrega de um projeto funcional sem comprometer suas características e objetivos principais.

\section{Análise dos resultados obtidos}

O desenvolvimento da aplicação BiteUp permitiu avaliar a viabilidade de uma ferramenta voltada ao acompanhamento nutricional aliado ao registro subjetivo de bem-estar, atendendo aos objetivos estabelecidos no início do projeto. De modo geral, os resultados demonstram que a solução foi capaz de entregar as funcionalidades essenciais para o monitoramento e acompanhamento diário de macronutrientes, além de oferecer ao usuário um meio intuitivo e rápido de registrar os efeitos de uma determinada refeição após a ingestão.

A aplicação foi projetada para oferecer uma interface intuitiva, agradável e de uso simplificado, permitindo que o usuário registre e visualize suas informações nutricionais de maneira fluida e eficiente. Ao selecionar um alimento, seus macronutrientes são automaticamente adicionados ao componente de acompanhamento nutricional presente na tela inicial, sem a necessidade de atualizar a página ou realizar qualquer procedimento adicional. As informações podem ser visualizadas tanto em formato numérico quanto por meio de barras de progresso, facilitando a compreensão do quanto já foi consumido e do que ainda falta para atingir a meta diária.

Outro destaque significativo da aplicação está no sistema de gerenciamento de refeições, que são cadastradas automaticamente com base no horário da inclusão. Essa funcionalidade elimina a necessidade de o usuário identificar manualmente cada refeição ou criar categorias repetitivas sempre que faz um novo registro, tornando o processo mais rápido e evitando interrupções na rotina.

Além disso, a aplicação disponibiliza um banco de alimentos pré-cadastrados, oferecendo praticidade para o registro das refeições. No entanto, reconhecendo a diversidade das dietas e das particularidades alimentares, o sistema também permite que o usuário crie seus próprios alimentos. Dessa forma, é possível registrar combinações específicas ou receitas pessoais como \textit{whey protein} misturado com outro ingrediente, definir as quantidades e salvar o item para uso rápido nos próximos dias, ampliando a flexibilidade e independência do usuário em relação ao banco interno, permitindo o acompanhamento de qualquer dieta, seja ela definida por alimentos ou receitas.

A aplicação também disponibiliza a \textit{view} 'daily' que desempenha o papel de exibir os alimentos registrados ao longo do dia. Nela, o usuário pode revisar suas escolhas alimentares, consultar detalhes nutricionais e excluir itens que tenham sido adicionados incorretamente, garantindo que o acompanhamento diário seja fiel, atualizado e totalmente sob controle do usuário.

Na execução de testes foi perceptível uma clareza no que se diz respeito às informações nutricionais e à agilidade nos processos, como o registro de alimentos. O cálculo automático de macronutrientes, aliado ao controle visual do progresso diário, contribui significativamente para uma percepção clara das metas nutricionais. A possibilidade de os usuários cadastrarem alimentos mostrou-se importante para tornar a aplicação personalizável ao mesmo tempo que enriquece progressivamente o banco de dados.

O sistema de \textit{feedback} implementado apresenta uma limitação inerente: a dependência do acúmulo de dados por parte do usuário, que alimenta o sistema ao efetuar avaliações sobre como se sentiu após determinadas refeições. Porém, quando dispõe de dados suficientes, o sistema demonstra capacidade de identificar correlações não óbvias, padrões de intolerância e sugerir \textit{insights} relevantes, podendo incluir recomendações como a troca do horário de consumo de algum alimento, a não ingestão ou até a substituição de um determinado alimento por outro. Dessa forma, o sistema é capaz de lidar com possíveis intolerâncias, substituindo alimentos problemáticos por alternativas com perfil nutricional semelhante, mantendo assim as metas estabelecidas.

Alguns desafios foram identificados ao longo do desenvolvimento. Entre eles, destacam-se limitações relacionadas à quantidade de dados disponíveis para análise, necessidade de desenvolver outras formas de interação com a IA, e a dependência do preenchimento do usuário para o acúmulo de dados.

% colocar imagem da tela inicial

\section{Trabalhos futuros}

Apesar dos resultados positivos alcançados com o desenvolvimento da aplicação, diversas oportunidades de aprimoramento podem ser exploradas em trabalhos futuros, ampliando significativamente sua eficiência, usabilidade e capacidade de personalização. Uma das direções mais promissoras diz respeito ao aperfeiçoamento do sistema de recomendação baseado em inteligência artificial. Em versões posteriores, a aplicação poderá empregar técnicas mais avançadas de aprendizado de máquina, testar novos modelos de coleta e estruturação de dados e implementar formas mais ricas de interação entre o usuário e a IA, tornando o processo de análise mais preciso e o retorno das recomendações mais alinhado ao perfil individual de cada pessoa.

No que diz respeito à experiência do usuário, há amplo espaço para evolução. A criação de uma tela dedicada ao perfil do usuário, contendo uma jornada de uso mais completa, gráficos mensais de desempenho, indicadores de progresso e contadores visuais de atingimento de metas, poderia enriquecer profundamente a navegação e reforçar o acompanhamento nutricional. A adoção de técnicas de gamificação, como medalhas, níveis, metas diárias e desafios semanais também se apresenta como uma estratégia eficaz para aumentar o engajamento contínuo, tornando o uso da aplicação mais motivador e prazeroso. 

Atualmente, os dados do usuário não podem ser alterados após o cadastro inicial, o que limita a flexibilidade e a adaptação do sistema a mudanças nos objetivos ou nas características pessoais. Trabalhos futuros podem incluir a possibilidade de edição segura desses dados, garantindo maior autonomia e precisão na personalização da experiência.

Outra frente interessante para aprimorar a experiência do usuário seria a implementação de funcionalidades que permitam alternar facilmente entre objetivos nutricionais. Por exemplo, um módulo de ciclo de \textit{bulking/cutting} possibilitaria que o usuário trocasse seus objetivos macronutricionais com apenas um clique, ajustando automaticamente a distribuição de proteínas, carboidratos e gorduras de acordo com a fase atual. Essa funcionalidade tornaria o acompanhamento mais dinâmico e flexível, atendendo às necessidades de usuários que seguem ciclos de ganho de massa e perda de gordura de forma estruturada, como atletas de musculação, proporcionando uma experiência mais completa, motivadora e alinhada às metas individuais de cada usuário.

O sistema também carece em módulos de segurança e privacidade que poderiam ser solucionados com implementação de criptografia avançada, autenticação multifatorial e protocolos de proteção. Tais medidas não apenas reforçam a segurança da informação, mas também aumentam a confiança do usuário ao lidar com dados sobre sua saúde, alimentação e bem-estar.








