% resumo no idioma do documento

\begin{abstract}
O crescente interesse por nutrição e bem-estar tem ampliado a demanda por ferramentas digitais que auxiliem no acompanhamento alimentar. Apesar da disponibilidade de diversas soluções tecnológicas, muitas pessoas ainda enfrentam dificuldades para monitorar adequadamente o consumo de macronutrientes ao longo do dia, comprometendo o alcance de seus objetivos nutricionais. Nesse contexto, este projeto propõe o desenvolvimento de uma aplicação móvel que permita ao usuário registrar os alimentos consumidos e acompanhar, em tempo real, tanto o impacto nutricional quanto os efeitos subjetivos dessas escolhas alimentares em seu bem-estar.

A aplicação realiza o cálculo automático da quantidade de macronutrientes consumidos e restantes, facilitando o planejamento alimentar diário e promovendo a adesão a hábitos mais saudáveis. Seu principal diferencial reside na capacidade de analisar a resposta individual do usuário após cada refeição, transcendendo o simples registro calórico e nutricional. Por meio de um sistema de \textit{feedback} recorrente, os usuários podem informar aspectos relacionados ao seu estado pós-prandial, tais como nível de energia, humor, disposição e eventuais desconfortos físicos.

Os dados coletados são processados para identificar padrões comportamentais e nutricionais, possibilitando inferências sobre quais alimentos contribuem para sensações positivas ou negativas. Com o acúmulo de informações, o sistema se torna progressivamente mais preciso em suas recomendações, adaptando-se dinamicamente às necessidades e preferências individuais de cada usuário. Além das funcionalidades de monitoramento nutricional e inventário de alimentos, a solução incorpora técnicas de aprendizado de máquina para oferecer uma experiência adaptativa e personalizada.

O projeto considera ainda desafios essenciais relacionados à precisão das informações nutricionais, à usabilidade da interface e à segurança dos dados pessoais. Dessa forma, busca-se desenvolver uma ferramenta eficiente, confiável e centrada na promoção de uma alimentação que contribua efetivamente para o equilíbrio físico e emocional dos indivíduos.
\end{abstract}
