\chapter[\hspace{1mm}]{- Requisitos de Usuários} 
\label{append:requistos}

\begin{table}[h]
\centering
\caption{Requisitos Não-Funcionais}
\label{tab:tabela_a1}
\begin{tabular}{
|>{\centering\arraybackslash}m{0.7cm}
|>{\raggedright\arraybackslash}m{8.5cm}
|>{\centering\arraybackslash}m{2.5cm}
|>{\centering\arraybackslash}m{2.5cm}|}
\hline
\textbf{N°} & \textbf{Descrição do Requisito} & \textbf{Categoria} & \textbf{Referência} \\ \hline

01
& Os dados em trânsito devem ser criptografados utilizando protocolo seguro.
& Segurança
& Art. 20 IN 134 \\ \hline

02
& Dados críticos, como senhas de usuários, devem ser armazenados criptografados no banco de dados.
& Segurança
& Art. 20 IN 134 \\ \hline

03
& O sistema deve permitir a realização de \textit{backups} periódicos.
& Segurança
& Art. 30 IN 134 \\ \hline

04
& A interface do sistema deve ser intuitiva, com elementos visuais que favoreçam a experiência do usuário.
& Usabilidade
& Art. 20 IN 134 \\ \hline

05
& A interface deve ser responsiva e acessível em dispositivos móveis.
& Usabilidade
& Art. 20 IN 134 \\ \hline

06
& O sistema deve ser capaz de operar com múltiplos usuários logados e requisições simultâneas.
& Escalabilidade
& Art. 20 IN 134 \\ \hline

07
& O sistema deve possibilitar a restauração dos dados em caso de falhas graves, sem comprometimento da integridade dos dados.
& Confiabilidade
& Art. 30 IN 134 \\ \hline

08
& A solução deve funcionar em modo \textit{offline}, garantindo continuidade de operação mesmo em ambientes sem conexão à rede.
& Disponibilidade
& Art. 20 IN 134 \\ \hline

09
& A solução deve estar em conformidade com os princípios do ALCOA++, garantindo a integridade, rastreabilidade e controle dos dados.
& Integridade de Dados
& Art. 20 IN 134 \\ \hline

\end{tabular}
\end{table}

\begin{table}[h]
\centering
\caption{Requisitos Funcionais}
\label{tab:tabela_a2}
\begin{tabular}{
|>{\centering\arraybackslash}m{0.7cm}
|>{\raggedright\arraybackslash}m{8.5cm}
|>{\centering\arraybackslash}m{2.5cm}
|>{\centering\arraybackslash}m{2.5cm}|}
\hline
\textbf{N°} & \textbf{Descrição do Requisito} & \textbf{Categoria} & \textbf{Referência} \\ \hline

01 
& O sistema deve possuir login para acesso, com diferentes grupos de usuários, sendo eles: Utilizador, Gestor e Administrador.
& Gestão de Acesso
& Art. 10 IN 134 \\ \hline

02
& Grupo Utilizador deve possuir privilégios apenas para acesso aos formulários para submissão e visualização de respostas.
& Gestão de Acesso
& Art. 36 IN 134 \\ \hline

03
& Grupo Gestor deve possuir privilégios para cadastro/gestão dos formulários e acesso de visualização de respostas.
& Gestão de Acesso
& Art. 10 IN 134 \\ \hline

04
& Grupo Administrador deve possuir privilégios para cadastro e gestão de usuários do sistema.
& Gestão de Acesso
& Art. 10 IN 134 \\ \hline

05
& Formulários devem possuir campos para preenchimento de informação de "Descrição", "Objetivo" e campos para cadastro das perguntas.
& Estrutura dos Formulários
& Art. 20 IN 134 \\ \hline

06
& Formulários devem conter registros relativos à sua criação e modificações, incluindo a identificação do usuário responsável pela ação e registro de data/hora.
& Estrutura dos Formulários
& Art. 33 IN 134 \\ \hline

07
& Formulários devem possuir um identificador exclusivo, informação de status (Elaboração / Ativo / Obsoleto / Cancelado), versão e histórico de revisão.
& Estrutura dos Formulários
& Art. 20 IN 134  \\ \hline

08
& Usuário do grupo Gestor deve poder alterar o status dos formulários de "Elaboração" para "Ativo" e de "Ativo" para "Cancelado".
& Estrutura dos Formulários
& Art. 20 IN 134  \\ \hline

09
& Deve ser possível editar os formulários com status de "Elaboração" ou "Ativo". Para formulário ativo, o sistema deve alterar o status para "Obsoleto" e criar uma nova versão do formulário. Nova versão do formulário deve possuir status "Elaboração", versão atualizada de modo incremental e histórico de revisão.
& Alteração dos Formulários
& Art. 20 IN 134  \\ \hline

10
& Somente formulários com status de "Ativo" devem estar disponíveis para submissão de respostas.
& Submissão de Respostas
& Art. 20 IN 134  \\ \hline

11
& O sistema deve possuir funcionalidade de trilha de auditoria, para registro detalhado de ações de cadastro e alterações, incluindo a identificação do objeto, ação realizada, valores alterados (antes e depois) e a identificação do usuário responsável pela ação e registro de data/hora.
& Trilha de Auditoria
& Art. 33 IN 134  \\ \hline
\end{tabular}
\end{table}

