% resumo no segundo idioma
% como parametros devem ser passados o titulo e as palavras-chave
% no outro idioma, separados por vírgulas

% \begin{englishabstract}{Using \LaTeX\ to Prepare Documents at TSI}{Electronic document preparation. \LaTeX. ABNT. IFSUL Câmpus Charqueadas}
% This document is an example on how to prepare documents at IFSul Câmpus Charqueadas. \emph{The text in
% the abstract should not contain more than 500~words.}
% \end{englishabstract}

\begin{englishabstract}{}{Data collection. Data Integrity. Pharmaceutical Industry. ALCOA}
Digital tools for electronic data collection, such as \textit{Google Forms}, are widely used in various contexts. However, in regulated sectors such as the pharmaceutical industry, the use of these tools presents limitations related to data integrity requirements and the feasibility of computerized system validation. Therefore, this work proposes the architecture and a prototype of the Track Forms platform, designed for the collection and storage of heterogeneous data in regulated pharmaceutical processes, aiming to ensure data integrity in compliance with the ALCOA++ model (Attributable, Legible, Contemporaneous, Original, Accurate, Complete, Consistent, Enduring, Available, and Traceable).

The solution was implemented using Java (Spring Boot) for the back-end and MongoDB for persistence (non-relational database). The RESTful architecture includes role-based access control, audit trail, metadata recording, support for dynamic forms with heterogeneous structures, and mechanisms for offline operation (PWA), ensuring traceability and preservation of the original record.

As a partial result, the Track Forms prototype enables the creation of dynamic forms, submission of responses, and persistence of information with metadata that supports the ALCOA++ attributes. Future work includes the implementation of temporal control for submissions, scheduled backups, and improvements to administrative functionalities.
\end{englishabstract}