\chapter{Cronograma e próximos passos}
\label{ch:conclusoes_trabalhos_futuro}

Os próximos passos do trabalho concentram-se na construção da camada de \textit{front-end}, responsável por consumir a \textit{API} já desenvolvida e realizar a integração com a camada de \textit{back-end}.

No escopo de desenvolvimento, está prevista a implementação de funcionalidades de controle de acesso e segurança, visando garantir a proteção dos dados e a restrição adequada às funcionalidades do sistema conforme os perfis de usuário.

Além disso, será realizada a adequação da lógica da aplicação para suportar o modelo \textit{Progressive Web Application} (PWA), com o objetivo de possibilitar o uso da aplicação em modo \textit{offline}.

Entre as melhorias previstas, destacam-se:
\begin{itemize}
    
    \item A introdução de controles temporais para a submissão de formulários, com o propósito de assegurar que os dados sejam registrados no banco de dados dentro de um intervalo de tempo definido após o preenchimento, atendendo ao atributo \textit{Contemporâneo} e reforçando a integridade dos registros.

    \item A implementação de uma funcionalidade para alteração de respostas de forma controlada e rastreável, com preservação do dado original. Essa abordagem permitirá a correção de informações quando necessário, sem comprometer a autenticidade e a rastreabilidade dos registros, em conformidade com os princípios de integridade dos dados.

    \item A implementação de funcionalidades voltadas à gestão administrativa, como o mecanismo de backup e restauração de dados, visando garantir a durabilidade, a disponibilidade e a segurança das informações armazenadas.
\end{itemize}

Essas ações visam aprimorar a robustez, a confiabilidade e a conformidade da aplicação, alinhando-a às melhores práticas de desenvolvimento e aos requisitos regulatórios aplicáveis.

\section{Cronograma}

\begin{table}[!h]
\centering
\caption{Cronograma}
\label{tab:cronograma}
\begin{tabular}{c|c|c|c|c|c|c|c|c|c|c|}
\cline{2-11}
\multicolumn{1}{l|}{}           & \multicolumn{10}{c|}{\textbf{TCC2- Semanas}} \\ \hline
\multicolumn{1}{|c|}{Atividade} &1 - 2&3 - 4 &5 - 6&7 - 8&9-10&11-12&13-14&15-16&17-18&19-20\\ \hline
\multicolumn{1}{|c|}{1}         &X    &X     &     &     &    &    &    &    &    &     \\ \hline
\multicolumn{1}{|c|}{2}         &     &      &     &X    &X   &X   &X   &X   &    &     \\ \hline
\multicolumn{1}{|c|}{3}         &     &X     &X    &X    &    &    &    &    &    &     \\ \hline
\multicolumn{1}{|c|}{4}         &X    &      &     &     &    &    &    &    &    &     \\ \hline
\multicolumn{1}{|c|}{5}         &X    &      &     &     &    &    &    &    &    &     \\ \hline
\multicolumn{1}{|c|}{6}         &     &      &     &X    &X   &X   &    &    &    &     \\ \hline
\multicolumn{1}{|c|}{7}         &     &      &     &     &    &X   &X   &X   &X   &    \\ \hline
\multicolumn{1}{|c|}{8}         &     &      &     &     &    &    &    &    &    &X  \\ \hline
\end{tabular}
\end{table}

\subsection{Detalhamento das atividades}

\begin{itemize}
    \item \textbf{Atividade 1 – Implementação das Funcionalidades de Segurança}: Implementar os mecanismos de controle de acesso e segurança para o sistema, garantindo a proteção dos dados e a restrição adequada conforme os perfis de usuário.
    \item \textbf{Atividade 2 – Desenvolvimento \textit{Front-end}}: Desenvolver a interface de usuário utilizando tecnologias adequadas, de forma a integrá-la eficientemente com a camada de \textit{back-end}, garantindo uma comunicação fluida e segura entre as diferentes partes da aplicação.
    \item \textbf{Atividade 3 – Implementação da Funcionalidade para Uso \textit{Off-line}}: Realizar a implementação da funcionalidade e a integração do sistema como uma \textit{Progressive Web App (PWA)}, possibilitando o uso em modo \textit{off-line}.
    \item \textbf{Atividade 4 – Melhoria - Controle Temporal para a Submissão de Respostas}: Implementar mecanismos que garantam o registro dos dados dentro de um intervalo de tempo definido após o preenchimento, assegurando a temporalidade e a autenticidade dos registros.
    \item \textbf{Atividade 5 – Melhoria - Implementação de Funcionalidade de Alteração de Respostas}: Desenvolver uma funcionalidade que permita a correção de respostas de forma controlada e rastreável, mantendo o dado original para fins de auditoria.
    \item \textbf{Atividade 6 – Melhoria - Implementação das Funcionalidades de Backup e Restauração de Dados}: Implementar funcionalidades que possibilitem ao administrador realizar backup e restauração de dados de forma segura e eficiente.
    \item \textbf{Atividade 7 – Revisão do texto e apresentação}: Revisar o documento do trabalho de conclusão de curso para adequar os resultados finais, incluir o tópico de trabalhos futuros e preparar a apresentação para a defesa do TCC.
    \item \textbf{Atividade 8 – Defesa do trabalho}: Realizar a apresentação final e defesa do TCC 2 perante a banca avaliadora.
\end{itemize}

% \section{Trabalhos Futuros}

% Sed ut perspiciatis unde omnis iste natus error sit voluptatem accusantium doloremque laudantium, totam rem aperiam, eaque ipsa quae ab illo inventore veritatis et quasi architecto beatae vitae dicta sunt explicabo. Nemo enim ipsam voluptatem quia voluptas sit aspernatur aut odit aut fugit, sed quia consequuntur magni dolores eos qui ratione voluptatem sequi nesciunt. Neque porro quisquam est, qui dolorem ipsum quia dolor sit amet, consectetur, adipisci velit, sed quia non numquam eius modi tempora incidunt ut labore et dolore magnam aliquam quaerat voluptatem. Ut enim ad minima veniam, quis nostrum exercitationem ullam corporis suscipit laboriosam, nisi ut aliquid ex ea commodi consequatur? Quis autem vel eum iure reprehenderit qui in ea voluptate velit esse quam nihil molestiae consequatur, vel illum qui dolorem eum fugiat quo voluptas nulla pariatur?

% At vero eos et accusamus et iusto odio dignissimos ducimus qui blanditiis praesentium voluptatum deleniti atque corrupti quos dolores et quas molestias excepturi sint occaecati cupiditate non provident, similique sunt in culpa qui officia deserunt mollitia animi, id est laborum et dolorum fuga. Et harum quidem rerum facilis est et expedita distinctio. Nam libero tempore, cum soluta nobis est eligendi optio cumque nihil impedit quo minus id quod maxime placeat facere possimus, omnis voluptas assumenda est, omnis dolor repellendus. Temporibus autem quibusdam et aut officiis debitis aut rerum necessitatibus saepe eveniet ut et voluptates repudiandae sint et molestiae non recusandae. Itaque earum rerum hic tenetur a sapiente delectus, ut aut reiciendis voluptatibus maiores alias consequatur aut perferendis doloribus asperiores repellat.


