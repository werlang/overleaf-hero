\chapter{Introdução}
\label{ch:introducao}

Ao longo da história, a coleta de dados tem desempenhado um papel fundamental na construção do conhecimento humano, servindo como base para decisões estratégicas em diversas áreas. No Brasil, essa prática pode ser identificada desde o século XIX, quando os primeiros levantamentos censitários foram realizados por autoridades locais, por meio de listas que discriminavam os habitantes residentes em seus distritos, com o objetivo de conhecer a composição populacional e subsidiar decisões de natureza política \cite{botelho2005censos}.

Nos séculos XX e XXI, a coleta de dados consolidou-se como uma ferramenta essencial para o desenvolvimento de pesquisas acadêmicas e científicas, bem como para a análise de fenômenos socioambientais, econômicos e culturais. Com o avanço da tecnologia, os métodos de pesquisa evoluíram significativamente, e o uso de ferramentas digitais para coleta e tratamento de dados passou a permitir uma compreensão mais ampla e ágil das informações. Esse progresso decorre da informatização e, frequentemente, da automação de etapas fundamentais do processo, como a coleta e o processamento de dados. Além de facilitar o armazenamento, essas inovações também viabilizam o compartilhamento e a análise das informações de forma mais eficiente. Tal evolução proporcionou maior agilidade na coleta e na interpretação das informações, tornando os resultados mais acessíveis e precisos para diversas áreas de estudo \cite{site:guiab2b2025}.

Exemplos relevantes de pesquisas e análises de dados realizadas nos últimos anos incluem estudos sobre a pandemia mundial do coronavírus (COVID-19), que viabilizaram o monitoramento da disseminação do vírus e da taxa de mortalidade, auxiliando na formulação de estratégias de saúde pública, conforme praticado no Brasil pelo Ministério da Saúde \cite{site:minSaudeCOVID2025}. Outro exemplo relevante são as pesquisas voltadas às mudanças climáticas, que possibilitam uma avaliação sobre as vulnerabilidades e os riscos de desastres ambientais, contribuindo para os estudos que visam mitigar os impactos e promover a conscientização ambiental \cite{syrkis2024vulnerabilidades}.

No setor farmacêutico, os dados desempenham um papel fundamental em diversas fases, desde os estudos clínicos voltados ao desenvolvimento de novos fármacos até as etapas da cadeia produtiva e da pós-comercialização, como fabricação, armazenamento, distribuição e monitoramento contínuo dos medicamentos. Nesses processos, os dados podem ser registrados de forma manual, digital ou híbrida \cite{ANVISARDC658}.

A autenticidade, confiabilidade e rastreabilidade das informações, sejam elas registradas de forma manual ou digital, são cruciais para garantir a segurança e a eficácia dos medicamentos, assegurando que estejam adequados para uso sem comprometer a saúde dos pacientes. Para atender a esses requisitos de qualidade, órgãos regulatórios como a ANVISA, no Brasil, e a FDA, nos Estados Unidos, estabelecem e normatizam padrões de Boas Práticas. No contexto da governança de dados, a principal atenção é direcionada à integridade de dados \cite{pics2021}.

Embora plataformas digitais de coleta de dados sejam práticas, acessíveis e amplamente utilizadas em outros setores e contextos, limitações relacionadas aos requisitos de integridade de dados no âmbito farmacêutico comprometem seu uso em processos regulados do setor.

\section{Objetivo}
\label{sec:objetivo}

O objetivo principal deste trabalho é propor uma solução para a coleta e o armazenamento de dados heterogêneos\footnote{Dados que apresentam diversos formatos e estruturas.} em processos regulados da indústria farmacêutica. Tais processos possuem exigências de integridade de dados e conformidade com padrões normativos que impossibilitam o uso de plataformas digitais genéricas de coleta de dados, como o \textit{Google Forms}.

\subsection{Objetivos Específicos}
\label{subsec:objespecificos}

O presente trabalho tem os seguintes objetivos específicos:
\begin{itemize}
    \item Fornecer um meio para coleta e gerenciamento centralizado de dados, facilitando o acesso, a análise e a organização de informações heterogêneas.
    \item Garantir o atendimento aos requisitos de integridade de dados, além dos aspectos necessários para a implementação da solução.
    \item Viabilizar a digitalização dos processos, reduzindo a utilização de documentos impressos no registro das informações e otimizando a avaliação dos dados.
\end{itemize}


\section{Organização do Trabalho}
\label{sec:organizacao}

O \hyperref[ch:referencial_teorico]{segundo capítulo --- Referencial Teórico}, aborda os conceitos de Integridade de Dados e Validação de Sistemas Computadorizados no contexto farmacêutico. Esses fundamentos são essenciais para a compreensão dos requisitos adotados no desenvolvimento da solução, bem como dos aspectos necessários à sua implementação.

No \hyperref[ch:trabalhos_relacionados]{terceiro capítulo --- Trabalhos Relacionados}, apresenta-se um estudo sobre trabalhos e aplicações relacionados à proposta deste projeto. A avaliação das soluções existentes foi conduzida com base em ferramentas de propósito semelhante, visando identificar lacunas e oportunidades de aprimoramento em relação aos objetivos estabelecidos.

O \hyperref[ch:solucao_proposta]{quarto capítulo --- Solução Proposta} apresenta a metodologia empregada no desenvolvimento do trabalho, detalhando as ferramentas tecnológicas utilizadas, os critérios que motivaram sua escolha e a contribuição de cada uma para a construção da solução.

Em seguida, o \hyperref[ch:resultados_discussoes]{ quinto capítulo --- Resultados Parciais}, traz uma síntese dos primeiros resultados com o protótipo desenvolvido, demonstrando o andamento do trabalho e avaliando a viabilidade da proposta frente aos objetivos definidos.

Por fim, o \hyperref[ch:conclusoes_trabalhos_futuro]{sexto capítulo --- Cronograma e Próximos Passos}, apresenta o planejamento para a conclusão do desenvolvimento do trabalho de conclusão de curso, detalhando as próximas atividades a serem executadas.